\def\dokumentTyp{Skript} 	


\def\hauptsprache{ngerman}
\def\andereSprachen{}
%%%%%%%%%%%%%%%%%%%%%%%%%%%%%%%%%%%%%%%%%%%%%%%%%%%%%%%%%%%%%
%   LaTeX-Arbeitsvorlage, Version 5.6                 	  	%
%   Autor: 		Mirco Lukas <http://mircol.de>              %
%   Lizenz: 	The MIT License (MIT)						%
%   Updates: 	https://github.com/MircoL/LatexTemplateDE   %
%%%%%%%%%%%%%%%%%%%%%%%%%%%%%%%%%%%%%%%%%%%%%%%%%%%%%%%%%%%%%

\RequirePackage{xstring}

\makeatletter
	\@ifundefined{dokumentTyp}{%
		\errmessage{DokumentTyp muss angegeben werden!} %
		\stop %
	}
\makeatother

\IfStrEqCase{\dokumentTyp}{
	{Skript}{\documentclass[12pt,ngerman,openany,numbers=noenddot,listof=totoc,bibliography=totoc,oneside]{scrbook}}%
	{Mitschrift}{\documentclass[a4paper,12pt,ngerman,oneside]{scrartcl}}%
	{Buch}{\documentclass[paper=a5,pagesize,footheight=-3cm]{scrbook}}%
	{Praesentation}{\documentclass{beamer}}%
}[\errmessage{Ungueltiger Dokumenttyp: \dokumentTyp}\stop]


% =====================================================
%    Kompatibilitätseinstellungen für ältere Versionen
% =====================================================
\providecommand{\hauptsprache}{ngerman}
\providecommand{\andereSprachen}{english,polish,russian}

% =====================================================
%    Pakete
% =====================================================

% sollte möglichst weit oben stehen
\usepackage{morewrites}

\usepackage{array}

\usepackage[xcolor,pdftex,rightbars]{changebar}

\usepackage{amsmath}

\usepackage{amssymb}

\usepackage[main=\hauptsprache,\andereSprachen]{babel}

% Tabelle farbig markieren: \columncolor{wunschfarbe} \rowcolor{}, \cellcolor{}
\usepackage{colortbl} 

\usepackage{enumitem}

\usepackage{environ}

\usepackage{esint}

\usepackage{eurosym}

\usepackage[OT2, T2A,TS1,T1]{fontenc}

\usepackage{graphicx}

\usepackage[utf8]{inputenc}

\usepackage{listings}

\usepackage{lipsum}

\usepackage{ifthen}

\usepackage{ifsym}

% erlaubt mehrseitige Tabellen:  \begin{longtable}{|l|r|c|p{2cm}|} \hline ... \end{longtable}
\usepackage{longtable} 

\usepackage{mVersion}
\increaseBuild

\usepackage{makeidx}

\usepackage{multicol}

% Workaround for Linux with texlive
\IfFileExists{oesch.sty} %
	{\usepackage{oesch}} %
	{%
		\errmessage{Warnung: 'oesch.sty' nicht gefunden. oeschfamily wird ein Synonym fuer 'em' sein.} %
	}
\providecommand{\oeschfamily}{\em}
\usepackage{pgfkeys}

\usepackage{polski}

% m u s s  vor hyperref stehen!
\usepackage{setspace}

\usepackage{soul}			

\usepackage{tcolorbox}

% Fußnoten durchgehend nummerieren
\usepackage{remreset}



\usepackage[\hauptsprache]{varioref}

\usepackage{wasysym}

\usepackage{xspace}

\usepackage{xargs}

\usepackage{yfonts}
% Lade Pakete abhängig vom DokumentTyp.
\IfStrEqCase{\dokumentTyp}{
	{Skript}{
		\usepackage[a4paper]{geometry}
		\usepackage[hyphens]{url}
		\usepackage[pdftex,pdfborderstyle={/S/U/W 1}]{hyperref}
	}
	{Mitschrift}{
		\usepackage{fancyhdr}
		\usepackage[a4paper]{geometry}
		\usepackage[hyphens]{url}
		\usepackage[pdftex,pdfborderstyle={/S/U/W 1}]{hyperref}
		\let\chapter\section
	}
	{Buch}{
		\usepackage[a5paper]{geometry}
		\usepackage{scrpage2}
		\usepackage[hyphens]{url}
		\usepackage[pdftex,pdfborderstyle={/S/U/W 1}]{hyperref}
	}
	{Praesentation}{
		
		\usecolortheme{whale}
		
		\newcommand{\fip}[1][]{\frametitle{\insertsection\xspace##1}\pause}
		\newcommand{\finp}[1][]{\frametitle{\insertsection\xspace##1}}
		
		\usefonttheme{professionalfonts} 
		
		\setbeamercovered{invisible}  % Overlays auf den Folien
		
		\setbeamertemplate{footline}{
			\begin{beamercolorbox}[sep=1.8em,wd=\paperwidth,leftskip=0.5cm,rightskip=0.5cm]{footlinecolor}
			\end{beamercolorbox}%
			\insertsectionnavigationhorizontal{\paperwidth}{}{\hskip0pt plus1filll} 
		}
		
		\setbeamertemplate{footline}{
			\begin{minipage}[c]{4cm}
				\get{Version 5/all/Titel}
			\end{minipage}
			\begin{minipage}[l]{4cm}
				\ifthenelse{\equal{\get{Version 5/Praesentationen/TitelKurz}}{}}{\get{Version 5/all/Titel}}{\get{Version 5/Praesentationen/TitelKurz}}
			\end{minipage}
			\begin{minipage}[c]{2cm}
				\StrSubstitute{\get{all/Autoren}}{&}{\par}
			\end{minipage}
			\hfill
			\begin{minipage}[c]{1.2cm}
				Folie \insertframenumber\ (\inserttotalframenumber)
			\end{minipage}
		}
	}
}

% m u s s   hier stehen!
\usepackage{todonotes}


% =====================================================
%    Eigene Befehle
% =====================================================
\newcommand{\andd}{\wedge}
	\newcommand{\bigandd}{\bigwedge}

%\newarray{Autoren}

\newcommand{\aufgabe}[1]{%
	\renewcommand{\aktuelleAufgabe}{#1}%
	\section*{Aufgabe \aktuelleAufgabe}%
}
	\newcommand{\teilaufgabe}[1]{\paragraph*{\underline{TA \aktuelleAufgabe.#1}}}
	\newcommand{\aktuelleAufgabe}{1}

\newboolean{isAppendixSet}
	\let\appendixOld\appendix
	\renewcommand{\appendix}{\ifthenelse{\boolean{isAppendixSet}}{}{\appendixOld\setboolean{isAppendixSet}{true}}}

\newcommand{\blitz}{\lightning\xspace}

\newcommand{\countAutoren}[1]{\newcounter{autorenC}\setcounter{autorenC}{#1}}

% Nicht \dh, sonst Kollision mit dem Buchstaben ð (Eth)
\newcommand{\dhe}{d.\,h.\xspace}
\newcommand{\Dhe}{D.\,h.\xspace}

\newcommand{\ellipse}{.\hspace*{-.1em}.\hspace*{-.1em}.}

\newcommand{\fallunterscheidung}[3][]{\ensuremath{#1 \left\{\begin{array}{cl}#2\\#3\end{array}\right.}}
	\newcommand{\fuer}{&\text{für }}
	\newcommand{\sonst}{&\text{sonst}}

\newcommand{\get}[1]{\pgfkeysvalueof{/VorlageVersion5/#1}}

\newcommand{\idr}{i.\,d.\,R.\xspace}
\newcommand{\Idr}{I.\,d.\,R.\xspace}

\newcommand{\length}[1]{\ensuremath{|#1|}}

\newcommand{\name}[1]{\textsc{#1}}

\newcommand{\neuerbegriff}[1]{\tcbox[size=small,on line,colframe=red!75!black,colback=blue!5!white,fontupper=\normalsize]{#1}\xspace}

\newcommand{\neuerbegriffIdx}[2][]{\ifthenelse{\equal{#1}{}}{\index{#2}\neuerbegriff{#2}}{\index{#1}\neuerbegriff{#2}}}

\makeindex

\newcommand{\obda}{o.\,B.\,d.\,A.\xspace}
\newcommand{\Obda}{O.\,B.\,d.\,A.\xspace}

\makeatletter
	\def\colorref@i		{\text{\textmarker{yellow}{(i)}}}
	\def\colorref@ii	{\text{\textmarker{green}{(ii)}}}
	\def\colorref@iii	{\text{\textmarker{red}{(iii)}}}
	\def\colorref@iv	{\text{\textcolor{white}{\textmarker{blue}{(iv)}}}}
\makeatother
\newcounter{colorrefCtr}
\def\colorref(#1){ %
	\setcounter{colorrefCtr}{#1} %
	\makeatletter %
		\csname colorref@\roman{colorrefCtr}\endcsname%
	\makeatother
}

\newcommand{\su}{s.\,u.\xspace}
\newcommand{\Su}{S.\,u.\xspace}

\newcommand{\orr}{\vee}
	\newcommand{\bigorr}{\bigvee}

\newcommand{\textmarker}[2]{\sethlcolor{#1}\hl{#2}}

\newcommand{\ua}{u.\,a.\xspace}
\newcommand{\Ua}{U.\,a.\xspace}

\newcommand{\va}{v.\,a.\xspace}
\newcommand{\Va}{V.\,a.\xspace}


\newcommand{\wiki}[1]{\href{#1}{Wikipedia}}

\newcommand{\zb}{z.\,B.\xspace}
\newcommand{\Zb}{Z.\,B.\xspace}

\newcommand{\zt}{z.\,T.\xspace}
\newcommand{\Zt}{Z.\,T.\xspace}


\IfStrEqCase{\dokumentTyp}{
	{Skript}{
		\newcommand{\qed}{\hfill\ensuremath{\square}}
	}
	{Mitschrift}{
		\newcommand{\qed}{\hfill\ensuremath{\square}}
	}
	{Buch}{
		\newcommand{\qed}{\hfill\ensuremath{\square}}
	}
	{Praesentation}{
		\newcommand{\itemz}{\item[\RIGHTarrow]}
	}%
}



% =====================================================
%    Eigene Umgebungen
% =====================================================
\newcommand{\bspT}{}
\newenvironmentx{beispiel}[2][1={}, 2={}]{%
	\renewcommand{\bspT}{#2}
	\paragraph{Beispiel\xspace#1}%
		\begin{quotation}%
			\setlength{\parindent}{0em}\noindent%
			\ifthenelse{\equal{\bspT}{o} \OR \equal{\bspT}{b}}{\vspace*{-2em}}{}%
}{%
	\ifthenelse{\equal{\bspT}{u} \OR \equal{\bspT}{b}}{\vspace*{-1em}}{}%
	\end{quotation}%
}

\newenvironmentx{beispiele}[2][1={}, 2={}]{%
	\renewcommand{\bspT}{#2}
	\paragraph{Beispiele\xspace#1}%
	\begin{quotation}%
		\setlength{\parindent}{0em}\noindent%
		\ifthenelse{\equal{\bspT}{o} \OR \equal{\bspT}{b}}{\vspace*{-2em}}{}%
	}{%
	\ifthenelse{\equal{\bspT}{u} \OR \equal{\bspT}{b}}{\vspace*{-1em}}{}%
	\end{quotation}%
}

\newenvironmentx{beweis}[2][1={}, 2={}]{%
	\renewcommand{\bspT}{#2}
	\paragraph{Beweis\xspace#1}%
	\begin{quotation}%
		\setlength{\parindent}{0em}\noindent%
		\ifthenelse{\equal{\bspT}{o} \OR \equal{\bspT}{b}}{\vspace*{-1em}}{}
	}{%
	\ifthenelse{\equal{\bspT}{u} \OR \equal{\bspT}{b}}{\vspace*{-1.3em}}{}%
	\qed%
	\ifthenelse{\equal{\bspT}{u} \OR \equal{\bspT}{b}}{\vspace*{.5em}}{}%
	\end{quotation}%
}

\newenvironment{enumeratealpha}{\begin{enumerate}[label={\alph*)}]}{\end{enumerate}}

\newenvironment{enumerateAlpha}{\begin{enumerate}[label={\Alph*)}]}{\end{enumerate}}

\newenvironment{enumerateroman}{\begin{enumerate}[label={\roman*)}]}{\end{enumerate}}

\newenvironment{enumerateRoman}{\begin{enumerate}[label={\Roman*)}]}{\end{enumerate}}


\newcommand{\tmpQx}{}
\newcommand{\kHand}[1]{%
	\IfStrEqCase{#1}{%
		{e}{e\hspace*{-.5em}\raisebox{.05em}{\reflectbox{\c{}}}\hspace*{-.2em}}%
		{a}{a\hspace*{-.6em}\raisebox{.05em}{\reflectbox{\c{}}}\hspace*{-.1em}}%
	}[\errmessage{Ogonek nur unter 'a' und 'e'!}\stop]	
}
\newcommand{\kFraktur}[1]{%
	\IfStrEqCase{#1}{%
		{e}{e\hspace*{-.1em}\raisebox{.05em}{\reflectbox{\c{}}}\hspace*{-.2em}}%
		{a}{a\hspace*{-.15em}\raisebox{.05em}{\reflectbox{\c{}}}\hspace*{-.15em}}%
	}[\errmessage{Ogonek nur unter 'a' und 'e'!}\stop]	
}
\newenvironmentx{quotex}[2][1={},2={}]{%
	\renewcommand{\tmpQx}{#1}
	\begin{quote}%
		\IfStrEqCase{#2}{
			{tt}{ \tt }%
			{hand}{ \let\k\kHand \oeschfamily }%
			{fraktur}{  \let\k\kFraktur \frakfamily }%
		}[ \normalfont ]
}{%
		\ifthenelse{\equal{\tmpQx}{}}{}{\par \mbox{} \hfill \normalfont --- \emph{\tmpQx}}%
	\end{quote}%
}
\newenvironmentx{quotationx}[2][1={},2={default}]{%
	\renewcommand{\tmpQx}{#1}
	\begin{quotation}%
		\IfStrEqCase{#2}{
			{tt}{ \tt }%
			{hand}{ \oeschfamily }%
			{fraktur}{ \frakfamily }%
			{default}{ \normalfont }%
		}[ \errmessage{Ungueltiger Parameter: \#2 muss 'tt' oder 'hand' oder 'fraktur' sein!}\stop ]
}{%
		\ifthenelse{\equal{\tmpQx}{}}{}{\par \mbox{} \hfill \normalfont --- \emph{\tmpQx}}%
	\end{quotation}%
}


% =====================================================
%    Farbige Boxen (tcolorbox) und Indizes
% =====================================================
\tcbuselibrary{listings,breakable}

\newcommand{\boxnummer}{\thetcbcounter}
\newtcolorbox[auto counter,number format=\arabic,list inside=blueboxes]{blueboxIdx}[2][box\boxnummer]{%
	colframe=blue!75!black,fonttitle=\bfseries,title={#2},phantomlabel={#1},breakable%
}

\newtcolorbox[auto counter,number format=\arabic,list inside=yellowboxes]{yellowboxIdx}[2][]{%
	colframe=yellow!99,fonttitle=\bfseries,title={\textcolor{black}{#2}},phantomlabel={#1},breakable%
}

\newtcolorbox[auto counter,number format=\arabic,list inside=greenboxes]{greenboxIdx}[2][]{%
	colframe=green,fonttitle=\bfseries,title={\textcolor{black}{#2}},phantomlabel={#1},breakable%
}

\newtcolorbox[number format=\arabic]{redbox}[1][]{%
	colframe=red!99,fonttitle=\bfseries,title={\textcolor{black}{Achtung!}},phantomlabel={#1},breakable%
}



% =====================================================
%    Listings einrichten
% =====================================================

\lstset{literate=
	{á}{{\'a}}1 {é}{{\'e}}1 {í}{{\'i}}1 {ó}{{\'o}}1 {ú}{{\'u}}1
	{Á}{{\'A}}1 {É}{{\'E}}1 {Í}{{\'I}}1 {Ó}{{\'O}}1 {Ú}{{\'U}}1
	{à}{{\`a}}1 {è}{{\`e}}1 {ì}{{\`i}}1 {ò}{{\`o}}1 {ù}{{\`u}}1
	{À}{{\`A}}1 {È}{{\'E}}1 {Ì}{{\`I}}1 {Ò}{{\`O}}1 {Ù}{{\`U}}1
	{ä}{{\"a}}1 {ë}{{\"e}}1 {ï}{{\"i}}1 {ö}{{\"o}}1 {ü}{{\"u}}1
	{Ä}{{\"A}}1 {Ë}{{\"E}}1 {Ï}{{\"I}}1 {Ö}{{\"O}}1 {Ü}{{\"U}}1
	{â}{{\^a}}1 {ê}{{\^e}}1 {î}{{\^i}}1 {ô}{{\^o}}1 {û}{{\^u}}1
	{Â}{{\^A}}1 {Ê}{{\^E}}1 {Î}{{\^I}}1 {Ô}{{\^O}}1 {Û}{{\^U}}1
	{œ}{{\oe}}1 {Œ}{{\OE}}1 {æ}{{\ae}}1 {Æ}{{\AE}}1 {ß}{{\ss}}1
	{ű}{{\H{u}}}1 {Ű}{{\H{U}}}1 {ő}{{\H{o}}}1 {Ő}{{\H{O}}}1
	{ç}{{\c c}}1 {Ç}{{\c C}}1 {ø}{{\o}}1 {å}{{\r a}}1 {Å}{{\r A}}1
	{€}{{\EUR}}1 {£}{{\pounds}}1 {~}{{\textasciitilde}}1
}

\definecolor{myGray}{RGB}{220,220,220}
\definecolor{darkspringgreen}{rgb}{0.09, 0.45, 0.27}

\lstset{
	basicstyle=\ttfamily\color{black}\small,
	keywordstyle=\bfseries\color{blue},
	identifierstyle=\underbar,
	commentstyle=\color{red},
	%showlines=true,
	backgroundcolor=\color{myGray},
	numbers=left,
	breakautoindent=true,
	stringstyle=\itshape\color{darkspringgreen},
	tabsize=4,
	frame=single,
	breaklines=true,
	postbreak=\raisebox{0ex}[0ex][0ex]{\ensuremath{\color{red}\hookrightarrow\space}},
	caption=\texttt\lstname
}



% =====================================================
%    Initiale Konfiguration
% =====================================================

\clubpenalty = 10000

\widowpenalty = 10000

\displaywidowpenalty = 10000

\makeatletter
	\@removefromreset{footnote}{chapter}
\makeatother

% Paragraphenüberschrifen immer mit einem Punkt beenden. Kann durch die Stern-Variante "\paragraph*{title}" unterdrückt werden.
\renewcommand\paragraphmark[1]{.} 

\let\epsilonOld\epsilon
\let\epsilon\varepsilon

\let\phiOld\phi
\let\phi\varphi


% =====================================================
%    Deklaration der Variablen
% =====================================================

\pgfqkeys{/VorlageVersion5}{
	all/Autoren/.initial					= {},
	all/Titel/.initial	 					= {},
	all/Untertitel/.initial 				= {},
	all/VersionPraefix/.initial 			= {},
	all/Version/.initial 					= {true},
	all/Icon/Breite/.initial				= {},
	all/Icon/URL/.initial					= {},
	all/Index/Boxen/blau/titel/.initial 	= {Liste der Sätze und Definitionen},
	all/Index/Boxen/blau/zeigen/.initial 	= {false},
	all/Index/Boxen/gelb/titel/.initial 	= {Liste der Beispiele},
	all/Index/Boxen/gelb/zeigen/.initial 	= {false},
	all/Index/Boxen/gruen/titel/.initial 	= {Liste der Fragen},
	all/Index/Boxen/gruen/zeigen/.initial 	= {false},
	all/Index/Begriffe/titel/.initial 		= {Index},
	all/Index/Begriffe/zeigen/.initial 		= {false},
	all/Index/Literatur/titel/.initial 		= {Literaturverzeichnis},
	all/Index/Literatur/zeigen/.initial 	= {false},
	Skript/AnmerkungenTitelseite/.initial 	= {},
	Mitschriften/Vorlesungsname/.initial 	= {},
	Mitschriften/Typ/.initial 				= {},
	Mitschriften/LfdNr/.initial 			= {},
	Mitschriften/Gruppe/.initial 			= {},
	Mitschriften/Headerhoehe/.initial		= {42pt},
	Buecher/Widmung/.initial				= {},
	Buecher/Modus/.initial					= {rl},
	Praesentationen/TitelKurz/.initial		= {},
	Praesentationen/Institut/.initial		= {},
	Praesentationen/InstitutKurz/.initial	= {}
}


% =====================================================
%    Generiere passende Titelseiten, Kopfzeilen etc.
% =====================================================

\AtBeginDocument{
	
	% Zulässige Werte testen
	\begingroup
		\newcommand{\checkbool}[1]{\ifthenelse{\equal{#1}{true} \OR \equal{#1}{false}}{}{\errmessage{Boolean '#1' muss 'true' oder 'false' sein!}\stop}}
		\checkbool{\get{all/Version}}
		\checkbool{\get{all/Index/Boxen/blau/zeigen}}
		\checkbool{\get{all/Index/Boxen/gelb/zeigen}}
		\checkbool{\get{all/Index/Boxen/gruen/zeigen}}
		\checkbool{\get{all/Index/Begriffe/zeigen}}
		\checkbool{\get{all/Index/Literatur/zeigen}}
		% ------------------------------------------------------------------------
		\ifx\documentTyp{Buch}
			\ifthenelse{\equal{Buecher/Modus}{e} \OR \equal{Buecher/Modus}{rl}}{}{\errmessage{//Buecher/Modus muss 'e' oder 'rl' sein!}\stop}
		\fi
	\endgroup
	
	\setVersion{\get{all/VersionPraefix}}
	
	\IfStrEqCase{\dokumentTyp}{
		{Skript}{%
			\thispagestyle{empty}
			\vspace*{4em}
			\begin{center}
				\ifthenelse{\equal{\get{all/Titel}}{}}{}{{\Huge \bf \get{all/Titel}\par}}
				\ifthenelse{\equal{\get{all/Untertitel}}{}}{}{\vspace*{.5em}{\large\bf\get{all/Untertitel}\par}}
			\end{center}
			\begin{flushright}
				\ifthenelse{\equal{\get{Skript/AnmerkungenTitelseite}}{}}{}{{\em \vspace*{1em}\get{Skript/AnmerkungenTitelseite}\par}}
				\ifthenelse{\equal{\get{all/Version}}{true}}{Version: \version, }{}\today\par\vspace*{1em}
				\StrSubstitute{\get{all/Autoren}}{&}{\par}
			\end{flushright}
			\begin{center}
				\IfFileExists{titel.jpg}{%
					\includegraphics[width=\get{all/Icon/Breite}\textwidth]{titel.jpg}\par%
					\ifthenelse{\equal{\get{all/Icon/URL}}{}}{}{{\tiny \href{\get{all/Icon/URL}}{Quelle}}}
				}{}
			\end{center}
			\newpage
			
			\pagenumbering{Roman}
			\newpage
			\tableofcontents 
			
			\newpage
			
			\newpage
			\pagenumbering{arabic} 
		}
		{Mitschrift}{
			\setlength{\parindent}{0cm}
			\setlength{\headheight}{\get{Mitschriften/Headerhoehe}}
			\pagestyle{fancy}
			\newcommand{\autorTemp}{\StrSubstitute{\get{all/Autoren}}{&}{\par}}
			\expandafter\lhead{\autorTemp}
			\chead{%
				\IfFileExists{titel.jpg}{\includegraphics[width=\get{all/Icon/Breite}\textwidth]{titel.jpg}\\}{}
				\ifthenelse{\equal{\get{Mitschriften/Gruppe}}{}}{}{Gruppe \get{Mitschriften/Gruppe}}
			}
			\rhead{\textbf{\emph{\get{Mitschriften/Vorlesungsname}}}\\\textbf{\get{Mitschriften/Typ}\ \get{Mitschriften/LfdNr}}\ifthenelse{\equal{\get{all/Version}}{true}}{\\Stand: \today}{}}
			\lfoot{} 
			\cfoot{-- \thepage\ --} 
			\rfoot{} 
			
			\global\long\def\headrulewidth{0.4pt}
			\global\long\def\footrulewidth{0.4pt}
			\ifthenelse{\equal{\get{all/Titel}}{}}%
				{\section*{\get{Mitschriften/Typ}\ \get{Mitschriften/LfdNr}}}%
				{\section*{\get{all/Titel}}}
			
			\ifthenelse{\equal{\get{all/Untertitel}}{}}{}{\subsubsection*{\get{all/Untertitel}}}
		}
		{Buch}{
			\pagestyle{scrheadings}
			\KOMAoptions{footinclude=true}
			\lehead{\thepage}
			\cehead{\leftmark}
			\rehead{}
			\lohead{}
			\cohead{\rightmark}
			\rohead{\thepage}
			\ofoot{}
			\cfoot{}
			\ifoot{}
			
			\renewcommand*\chapterpagestyle{empty}
		
			\KOMAoptions{twoside=false}
			
			\thispagestyle{empty}
			\vspace*{15em}
			\begin{center}
				\ifthenelse{\equal{\get{all/Titel}}{}}{}{{\Huge \get{all/Titel}} \par\vspace*{1em}}
				\ifthenelse{\equal{\get{all/Untertitel}}{}}{}{{\Large \get{all/Untertitel}} \par\vspace*{1em}}
			\end{center}
			\StrSubstitute{\get{all/Autoren}}{&}{\tabularnewline}[\autorTemp]
			\begin{flushright}
				\begin{tabular}r \autorTemp \end{tabular}
				
				\ifthenelse{\equal{\get{all/Version}}{true}}{Version \version}{}
			\end{flushright}
			
			\begin{center}
				\IfFileExists{titel.jpg}{\includegraphics[width=\get{all/Icon/Breite}\textwidth]{titel.jpg}}{}
			\end{center}
			
			\newpage
			
			
			\ifthenelse{\equal{\get{Buecher/Widmung}}{}}{}{
				\newpage
				\thispagestyle{empty}
				\mbox{}
				\newpage
				\thispagestyle{empty}
				\vspace*{18em}
				
				\begin{center}
					\textit{\get{Buecher/Widmung}}	
				\end{center}
				\newpage
			}
			\thispagestyle{empty}
			\mbox{}
			\tableofcontents
			
			\IfStrEqCase{\get{Buecher/Modus}}{
				{rl}{	
					\KOMAoptions{twoside=true}
				}%
				{e}{
					\KOMAoptions{twoside=false}
				}%
			}[\errmessage{Ungueltiger Modus, erlaubt sind ausschließlich 'rl' oder 'e'}\stop]
			
			\renewcommand{\chaptermark}[1]{\markboth{##1}{}}
			\renewcommand{\sectionmark}[1]{\markright{##1}}
		}
		{Praesentation}{
			\title{\get{all/Titel}}
			\subtitle{\get{all/Untertitel}}
			\StrSubstitute{\get{all/Autoren}}{&}{\par}[\autorTemp]
			\author{\autorTemp}
			\IfFileExists{titel.jpg}{\logo{titel.jpg}}{}
			\ifthenelse{\equal{\get{Praesentationen/Institut}}{}}{}{\institute{\get{Praesentationen/Institut}}}
			
			\frame{\titlepage}
			
			\frame{
				\finp
				\tableofcontents
			}
		}
	}
}

\AtEndDocument{
	\appendix
	
	\ifthenelse{\equal{\get{all/Index/Literatur/zeigen}}{true}}%
	{%
		\renewcommand{\bibname}{\get{all/Index/Literatur/titel}}
		\bibliographystyle{alphadin}
		\nocite{*}
		\bibliography{quellen}
	}%
	{}
	
	\IfStrEqCase{\dokumentTyp}{
		{Skript}{%
			\ifthenelse{\equal{\get{all/Index/Boxen/blau/zeigen}}{true}}{\tcblistof[\chapter]{blueboxes}{\get{all/Index/Boxen/blau/titel}}}{}
			
			\ifthenelse{\equal{\get{all/Index/Boxen/gelb/zeigen}}{true}}{\tcblistof[\chapter]{yellowboxes}{\get{all/Index/Boxen/gelb/titel}}}{}
		
			\ifthenelse{\equal{\get{all/Index/Boxen/gruen/zeigen}}{true}}{\tcblistof[\chapter]{greenboxes}{\get{all/Index/Boxen/gruen/titel}}}{}
		
			\ifthenelse{\equal{\get{all/Index/Begriffe/zeigen}}{true}}{
				\renewcommand{\indexname}{\get{all/Index/Begriffe/titel}}
				\addcontentsline{toc}{chapter}{\indexname}
				\printindex
			}{}
		}
		{Mitschrift}{%
			\ifthenelse{\equal{\get{all/Index/Boxen/blau/zeigen}}{true}}{\tcblistof[\chapter]{blueboxes}{\get{all/Index/Boxen/blau/titel}}}{}
			
			\ifthenelse{\equal{\get{all/Index/Boxen/gelb/zeigen}}{true}}{\tcblistof[\chapter]{yellowboxes}{\get{all/Index/Boxen/gelb/titel}}}{}
			
			\ifthenelse{\equal{\get{all/Index/Boxen/gruen/zeigen}}{true}}{\tcblistof[\chapter]{greenboxes}{\get{all/Index/Boxen/gruen/titel}}}{}
			
			\ifthenelse{\equal{\get{all/Index/Begriffe/zeigen}}{true}}{
				\renewcommand{\indexname}{\get{all/Index/Begriffe/titel}}
				\addcontentsline{toc}{chapter}{\indexname}
				\printindex
			}{}
		}
		{Buch}{%
			\ifthenelse{\equal{\get{all/Index/Boxen/blau/zeigen}}{true}}{\tcblistof[\chapter]{blueboxes}{\get{all/Index/Boxen/blau/titel}}}{}
			
			\ifthenelse{\equal{\get{all/Index/Boxen/gelb/zeigen}}{true}}{\tcblistof[\chapter]{yellowboxes}{\get{all/Index/Boxen/gelb/titel}}}{}
			
			\ifthenelse{\equal{\get{all/Index/Boxen/gruen/zeigen}}{true}}{\tcblistof[\chapter]{greenboxes}{\get{all/Index/Boxen/gruen/titel}}}{}
			
			\ifthenelse{\equal{\get{all/Index/Begriffe/zeigen}}{true}}{
				\renewcommand{\indexname}{\get{all/Index/Begriffe/titel}}
				\addcontentsline{toc}{chapter}{\indexname}
				\printindex
			}{}
		}
		{Praesentation}{%
			\ifthenelse{\equal{\get{all/Index/Begriffe/zeigen}}{true}}{
				\newenvironment{theindex}{\let\item\par}{}
				
				\newcommand\indexspace{}
				
				\begin{frame}
					\frametitle{\get{all/Index/Begriffe/titel}}
					\printindex
				\end{frame}
			}{}
		}
	}
}



\pgfqkeys{/VorlageVersion5}{
	all/Autoren						= {Mirco Lukas},
	all/Titel	 					= {Dokumentation zur \LaTeX-Vorlage},
	all/Untertitel 					= {Version~V5.2},
	all/VersionPraefix 				= {v5},
	all/Version 					= {false},
	all/Icon/Breite					= {1},
	all/Icon/URL					= {http://www.el-celta.de/latex2.gif},
	all/Index/Boxen/blau/zeigen		= {false},
	all/Index/Boxen/blau/titel 		= {Liste der Sätze und Definitionen},
	all/Index/Boxen/gelb/zeigen		= {false},
	all/Index/Boxen/gelb/titel 		= {Liste der Beispiele},
	all/Index/Boxen/gruen/zeigen	= {false},
	all/Index/Boxen/gruen/titel 	= {Liste der ...},
	all/Index/Begriffe/zeigen 		= {true},
	all/Index/Begriffe/titel 		= {B\ \ Index},
	all/Index/Literatur/zeigen	 	= {false},
	all/Index/Literatur/titel 		= {Literaturverzeichnis},
}


\usepackage{pdfpages}
\newcommand{\incpdf}[1]{\includepdf[pages=-,frame= true,delta=3mm 3mm]{#1}}

\begin{document}
%		Dieses Dokument ist die Doku zur \LaTeX-Vorlage von \name{Mirco Lukas}. Sie listet insbesondere selbstdefinierte Befehle und Umgebungen auf.

	% ------------------------------------------------------------------------------------------------------------------
	
	\chapter{Grundeinstellungen für Dokumente}
		\section{Der \texttt{\textbackslash dokumentTyp}}
			\index{dokumentTyp@\texttt{dokumentTyp}}
			Es gibt aktuell vier DokumentTypen. Diese führen zu verschiedenen Dokumentausgaben.
			\begin{tabbing}
				mmmmmmmmmmmmmmm				\= mmmmmmmmmmmmmmmm 				 \kill
				\bf Der \texttt{dokumentTyp}\ellipse 	
											\> \bf \ellipse erzeugt die \texttt{documentclass} \\ 
				\tt Skript					\> \tt scrbook\\
				\tt Mitschrift				\> \tt article \\
				\tt Buch					\> \tt scrbook \\
				\tt Praesentation			\> \tt beamer \\
				\emph{(ein anderer)}		\> \em (Fehlermeldung: Ungueltiger Dokumenttyp.)
			\end{tabbing}
		Die einzelnen Dokumente, die erzeugt werden, sehen wie folgt aus:
		
		
		\begin{center}
			\vspace*{20em}
			\subsection{Beispiel zu \tt Skript}
		\end{center}
		\incpdf{muster-skript.pdf}
		\newpage
		
		\begin{center}
			\vspace*{20em}
			\subsection{Beispiel zu \tt Mitschrift}
		\end{center}
		\incpdf{muster-mitschrift.pdf}
		\newpage
		
		\begin{center}
			\vspace*{20em}
			\subsection{Beispiel zu \tt Buch}
		\end{center}
		\incpdf{muster-buch.pdf}
		\newpage
		
		\begin{center}
			\vspace*{20em}
			\subsection{Beispiel zu \tt Praesentation}
		\end{center}
		\incpdf{muster-praesentation.pdf}
		\newpage


	
	\section{Dokumenteinstellungen}
		\subsection{Einstellungen vor dem Einbinden der Datei \texttt{mircol-v5.tex}}
			Die folgenden Einstellungen beeinflussen die Pakete \texttt{babel} und \texttt{varioref}:
			\begin{tabbing}
				mmmmmmmmmmm 			\= m \kill
				\bf Befehl				\> \bf Erklärung \\
				\verb|\hauptsprache|	\> Die Hauptsprache des Dokuments, \zb \verb|ngerman|, \verb|english|, \\
										\> \verb|polish|, \verb|russian|  \\
				\verb|\andereSprachen|	\> eine Liste weiterer Sprachen, die im Dokument vorkommen \\
										\> (durch Kommata getrennt)
			\end{tabbing}
		\subsection{Arrays}
			Es werden folgende Arrays definiert. Felder werden \emph{ohne Leerzeichen} mit "`\&"' getrennt: \verb|all/Autoren = {Alice&Bob}|.
			\begin{tabbing}
				mmmmmmmmmmmm 		\= mmmmmmmmmmmmmmmm				\=	mmmmmmmmmm	\kill
				\bf Name des Arrays	\> \bf Inhalt 					\> \bf Dokumentklassen\\
				 all/Autoren		\> Die Autoren der Dokumente	\> alle
			\end{tabbing}
		\subsection{Pgfkeys}
			Im Folgenden werden die existierenden Schlüssel-Werte-Paare gelistet. Die Dokumentklassen sind aus dem Schlüssel direkt ersichtlich. Dabei bedeuten:
			\begin{description}
				\item[string:] 
						Ein beliebiger Text. Zeilenumbrüche sind mit \verb|\par| zu definieren.
				\item[boolean:] 
						Genau eines der Werte \texttt{true} oder \texttt{false}.
				\item[integer:]
						Eine positive, ganze Zahl ($n \in \mathbb N^+ = \{1, \, 2, \, \ldots\}$)
				\item[double:]
						Eine positive Zahl ($n \in \mathbb Q^+$)
				\item[length:]
						Eine Zahl plus Einheit, \zb "`12em"'; "`42pt"'
				\item[url:]
						Eine Internetadresse.
				\item[{\tt \{\ldots\}}:] 
						Genau einer der genannten Werte.
			\end{description}
			\newpage
			\begin{tabbing}
				mmmmmmmmmmmmmmmmmm				\= mmmmmmmmmmmmmmmmmmmmmm 				\=\kill
				\bf Schlüssel					\> \bf Bedeutung 						\> \bf Wertebereich	\\
				all/Titel	 					\> Titel der Arbeit						\> string			\\
				all/Untertitel 					\> Untertitel							\> string			\\
				all/VersionPraefix 				\> Präfix der Versionsangabe			\> string			\\
				all/Version 					\> Anzeigen der Version					\> boolean 			\\
				all/Icon/Breite					\> Breite des Titelbildes				\> double 			\\
				all/Icon/URL					\> URL des Titelbildes					\> url				\\
				all/Index/Boxen/blau/zeigen 	\> Zeige den Index aller blauen Boxen	\> boolean 			\\
				all/Index/Boxen/blau/titel 		\> Titel des Index' aller blauen Boxen	\> string			\\
				all/Index/Boxen/gelb/zeigen 	\> Zeige den Index aller gelben Boxen	\> boolean 			\\
				all/Index/Boxen/gelb/titel 		\> Titel des Index' aller gelben Boxen	\> string			\\
				all/Index/Boxen/gruen/zeigen 	\> Zeige den Index aller grünen Boxen	\> boolean 			\\
				all/Index/Boxen/gruen/titel 	\> Titel des Index' aller grünen Boxen	\> string			\\
				all/Index/Begriffe/zeigen 		\> Zeige den Index der Begriffe			\> boolean			\\
				all/Index/Begriffe/titel 		\> Titel des Index' der Begriffe		\> string			\\
				all/Index/Literatur/zeigen		\> Zeige das Literaturverzeichnis		\> boolean			\\
				all/Index/Literatur/titel		\> Titel des Literaturverzeichnisses	\> string			\\
				Skript/AnmkerkungenTitelseite 	\> Eine Anmerkung, die auf der Titelseite\> string			\\ 
												\> erscheint												\\
				Mitschriften/Vorlesungsname 	\> Name der Vorlesung 					\> string			\\
				Mitschriften/Typ 				\> \zb \emph{Übung, Vorlesung, Praktikum}	
																						\> string			\\
				Mitschriften/LfdNr 				\> \zb die jeweilige Übungsnummer		\> string (!)		\\
				Mitschriften/Gruppe 			\> \zb die Übungsgruppe					\> string			\\
				Mitschriften/Headerhoehe		\> Höhe des Headers (falls Text hineinragt)
																						\> length			\\
				Buecher/Widmung					\> Eine Widmung							\> string			\\
				Buecher/Modus					\> Aufteilung in rechte und linke Seite (\texttt{rl})
																						\> \{\texttt{rl}, \texttt{e}\} \\
												\> oder jede Seite einzeln und zentriert (\texttt{e})					\\
				Praesentationen/TitelKurz		\> Ein Kurztitel (für die Fußzeile)		\> string			\\
				Praesentationen/Institut		\> Das Institut (für die Titelseite)	\> string			\\
				Praesentationen/InstitutKurz	\> Abkürzung des Instituts 				\> string 			\\
												\> (für die Fußzeile)
			\end{tabbing}
			\textbf{Besonderheiten:}
			\begin{itemize}
				\item Als Titelbild wird die Datei \texttt{titel.jpg} verwendet. Ist sie nicht vorhanden, bleiben die dazugehörigen Befehle ohne Wirkung.
				\item Als Literaturliste wird die Datei \texttt{quellen.bib} verwendet.
			\end{itemize}


	% ------------------------------------------------------------------------------------------------------------------
	

	\chapter{Eigene Befehle}
		\section{Allgemeine Befehle}
			Ein Stern gibt an, dass der Parameter optional ist. 
			\begin{tabbing}
				mmmmmmmmmmm			\= mmmmmmmmmmmmmmm							\= mmmmmmmmmm			\= \kill
				\bf Befehl			\> \bf Beschreibung							\> \bf Ergebnis 		\> \bf Parameter 				\\
				\verb|\andd|		\> Ein Synonym für \verb|\wedge|			\> $\andd$ 				\> ---							\\
				\verb|\aufgabe|		\> Nützlich für Übungsblätter				\> Aufgabe \#1 			\> \#1: Nr. der Aufgabe 		\\
				\verb|\bigandd|		\> Ein Synonym für \verb|\bigwedge|			\> $\bigandd$ 			\> --- 							\\
				\verb|\bigorr|		\> Ein Synonym für \verb|\bigvee|			\> $\bigorr$ 			\> --- 							\\
				\verb|\blitz|		\> Ein Synonym für							\> \lightning\xspace 	\> ---							\\
									\> \verb|\lightning\xspace|																			\\
				\verb|\colorref(i)|	\> Farbige Referenzzeichen					\> $i=1:$ \colorref(1)	\> $i \in \{1, \, \ldots, \, 4\}$	\\
									\>(\zb für Beweise)							\> $i=2:$ \colorref(2)										\\
									\>											\> $i=3:$ \colorref(3)										\\
									\>											\> $i=4:$ \colorref(4)										\\
				\verb|\ellipse|		\> Eine Ellipse	 (\href{https://de.wikipedia.org/wiki/Auslassungspunkte}{Wikipedia})
									\> \ellipse				\> ---							\\
				\verb|\fallunterscheidung|\\
									\> Für Fallunterscheidungen					\> Anm. a)				\> \#1*: Ein Präfix.			\\
									\>											\>						\> \#2: Der obere Fall			\\
									\>											\>						\> \#3: Der untere Fall			\\
				\verb|\fuer|		\> Für Fallunterscheidungen					\> Anm. a)				\> ---							\\
				\verb|\length|		\> Kurz für $|$Text$|$						\> \length{Text}		\> \#1: ein Text				\\
				\verb|\name|		\> Ein Text in Kapitälchen					\> \name{Der Autor}		\> \#1: ein Name				\\
				\verb|\neuerbegriff|\> Einführung eines Begriffs				\> Siehe Abschn. \ref{begriffsdefinitionen}				\\
				\verb|\neuerbegriffIdx| 
									\>Einführung eines Begriffs					\> Siehe Abschn. \ref{begriffsdefinitionen}				\\
				\verb|\orr|			\> Ein Synonym für \verb|\vee|				\> $\orr$ 				\> ---							\\
				\verb|\sonst|		\> Für Fallunterscheidungen					\> Anm. a)				\> ---							\\
				\verb|\teilaufgabe|	\> Nützlich für Übungsblätter				\> TA \verb|\aufgabe|.\#1 	
																										\> \#1: Nr. der Teilaufgabe 	\\
				\verb|\textmarker|	\> Ein Textmarker							\> \textmarker{green}{Text}
																										\> \#1: Farbe des Markers		\\
									\>											\>						\> \#2: Der zu mark. Text 		\\
				\verb|\qed|			\> Eine Box am Ende der Zeile				\> Anm. b)				\> ---							\\
				\verb|\wiki|		\> Ein Link zu Wikipedia					\>\wiki{https://de.wikipedia.org/wiki/LaTeX}
																										\> \#1: Eine Wiki-URL			\\
									\>  (hier \texttt{https://de.wikipedia.org/wiki/LaTeX})
			\end{tabbing}
			\newpage
			\textbf{Anmerkungen:}
			\begin{enumeratealpha}
				\item \verb|\fallunterscheidung[\operatorname{abs}(x) =]{x \fuer x \geq 0,}{-x \sonst.}|
						\fallunterscheidung[\operatorname{abs}(x) =]{x \fuer x \geq 0,}{-x \sonst.}
				\item \verb|Dieser Satz ist falsch. \qed| \\
						Dieser Satz ist falsch. \qed
			\end{enumeratealpha}
		\section{Umdefinition vorhandener Befehle}
			Die folgenden \TeX-/\LaTeX-Befehle wurden umdefiniert. Der Befehl \verb|\relax| deaktiviert ein Kommando.
			\begin{tabbing}
				mmmmmm			\= mmmmmmmmmmm			\= \kill
				\bf Befehl 		\> \bf Neue Bedeutung	\> \bf Bemerkung																										\\
				\verb|\appendix|\> \verb|\appendix|		\> Stellt nun sicher, dass der \LaTeX-Befehl 																			\\
								\> 						\> \verb|\appendix| nur einmal aufgerufen wird.																			\\
				\verb|\section|	\> \verb|\chapter|		\> nur in \texttt{Mitschrift}en, da dort keine \texttt{chapter} existieren.												\\
				\verb|\epsilon|	\> $\epsilon$			\> $\epsilonOld$ = \verb|\epsilonOld|																					\\
				\verb|\phi|		\> $\phi$				\> $\phiOld$ = \verb|\phiOld|
			\end{tabbing}
		\section{Gängige Abkürzungen}
			Die folgenden Befehle geben die Abkürzung aus, für die sie stehen, wobei zwischen den Wörtern ein Achtelgeviert-Abstand eingefügt wird.
			\begin{tabbing}
				mmmmmmm			\= mmmmmmmmmmmmmm						\= \kill
				Vergleiche		\>\texttt{o.$\sim$B.$\sim$d.$\sim$A.}:	\> Es sei o.~B.~d.~A. $x>0$.\\
				und				\>\verb|\obda|=\verb|o.\,b.\,d.\,a.|:	\> Es sei \obda $x>0$.
			\end{tabbing}
			Alle Befehle gibt es auch mit Großbuchstaben (\verb|\zB| -- \verb|\ZB|) am Anfang, etwa für Satzanfänge.
			\begin{tabbing}
				mmmmmmmmm						\= mmmmmmmmmmmmmmmmmmmmm		\=\kill
				\bf Befehl						\> \bf Bedeutung 				\> \bf Bemerkung\\
				\verb|\dhe|, \verb|\Dhe|		\> das heißt 					\> \verb|\dh| ist schon für \dh\ belegt.	\\
				\verb|\idr|, \verb|\Idr|		\> in der Regel															\\
				\verb|\obda|, \verb|\Obda|		\> ohne Beschränkung der Allgemeinheit 									\\
				\verb|\su|, \verb|\Su|			\> siehe unten															\\
				\verb|\ua|, \verb|\Ua|			\> unter anderem														\\
				\verb|\zb|, \verb|\Zb|			\> zum Beispiel															\\
				\verb|\zt|,	\verb|\Zt|			\> zum Teil				
			\end{tabbing}
		\section{Begriffsdefinitionen}
			\label{begriffsdefinitionen}
			Der Befehl \verb|\neuerbegriff{}| kennzeichnet einen Begriff durch Umrahmung:
			\begin{quotex}[\href{https://de.wikipedia.org/wiki/Betriebssystem}{Wikipedia}]
				Ein \neuerbegriff{Betriebssystem} ist eine Zusammenstellung von Computerprogrammen, die die Systemressourcen eines Computers wie Arbeitsspeicher, Festplatten, Ein- und Ausgabegeräte verwaltet und diese Anwendungsprogrammen zur Verfügung stellt. 
			\end{quotex}
			Der Befehl \verb|\neuerbegriffIdx{}| funktioniert wie folgt:
			\begin{itemize}
				\item Gibt man nur einen Parameter an, so erscheint der Text an der Stelle des Befehls und der gleiche Text steht im Index: \verb|\neuerbegriffIdx{Betriebssystem}| gibt \texttt{Betriebssystem} aus und schreibt auch \texttt{Betriebssystem} in den Index.
				\item Der optionale Parameter bietet die Möglichkeit, einen abweichenden Text in den Index zu schreiben. Dies ist nützlich für Index-Befehle oder falls der Begriff flektiert ist:
					\begin{quote}
						\verb|\neuerbegriffIdx[Betriebssystem, Definition]{Betriebssysteme}| sind \verb|\ellipse|
					\end{quote}
				schreibt \neuerbegriff{\texttt{Betriebssysteme}} \texttt{sind\ldots} anstelle des Befehls in den Text und \texttt{Betriebssystem, Definition} in den Index.
			\end{itemize}


	% ------------------------------------------------------------------------------------------------------------------


	\chapter{Eigene Umgebungen}
		\section{Übersicht}
			Die Umgebungen werden mit \verb|\begin{name} ... \end{name}| umschlossen. Ein Stern gibt an, dass der Parameter optional ist. Beispiele siehe unten.
			\begin{tabbing}
				mmmmmmmmmm				\= mmmmmmmmmmmmmmmmmmm					\= mmmmmmmmmmm									\kill
				\bf Befehl				\> \bf Beschreibung						\> \bf Parameter 								\\
				\verb|beispiel|			\> Eine \textbf{Beispiel}-Umgebung 		\> \#1${}^*$: Text hinter \emph{Beispiel} 		\\
										\>										\> \#2${}^*$: $x \in \{o, \, u, \, b\}$ (Abschn. \ref{abstaende beweis beispiel}) \\	
				\verb|beispiele|		\> Eine \textbf{Beispiele}-Umgebung 	\> \#1${}^*$: Text hinter \emph{Beispiele} 	\\
										\>										\> \#2${}^*$: $x \in \{o, \, u, \, b\}$ (Abschn. \ref{abstaende beweis beispiel}) \\	
				\verb|beweis|			\> Eine Umgebung, die mit 				\> \#1${}^*$: Text hinter \emph{Beweis}		\\
										\> dem \emph{qed}-Symbol ($\Box$) endet	\> \#2${}^*$: $x \in \{o, \, u, \, b\}$ (Abschn. \ref{abstaende beweis beispiel})		\\
				\verb|enumeratealpha|	\> Eine Auflistung mit a), b), \ellipse	\> ---											\\
				\verb|enumerateAlpha|	\> Eine Auflistung mit A), B), \ellipse \> ---											\\
				\verb|enumerateroman|	\> Eine Auflistung mit i), ii), \ellipse\> ---											\\
				\verb|enumerateRoman|	\> Eine Auflistung mit I), II), \ellipse\> ---											\\
				\verb|quotex|			\> Eine \texttt{quote}-Umgebung mit Extras
																		\> Siehe Abschnitt \ref{quotex_quotationx}.			\\
				\verb|quotationx|		\>	Analog zu \texttt{quotex}	\> Siehe Abschnitt \ref{quotex_quotationx}.
			\end{tabbing}
			
			\fbox{
				\begin{minipage}[t]{.9\textwidth}
					\texttt{\textbackslash begin\{enumeratealpha\} \\ \hspace*{2em}
						\textbackslash item ein Item \textbackslash item ein Item  \\
						\textbackslash end\{enumeratealpha\}}
					\begin{enumeratealpha} \item ein Item \item ein Item \end{enumeratealpha}
				\end{minipage}	
			} \par
			\fbox{
				\begin{minipage}[t]{.9\textwidth}
					\texttt{\textbackslash begin\{beispiel\}  \\ \hspace*{2em}
						\textbackslash lipsum[3] \\
					\textbackslash end\{beispiel\}}
					\begin{beispiel} \lipsum[3] \end{beispiel}
				\end{minipage}
			}
			\par
			\fbox{
				\begin{minipage}[t]{.9\textwidth}
					\texttt{\textbackslash begin\{beispiele\}[zur Vorlesung]  \\ \hspace*{2em}
						\textbackslash lipsum[3] \\
					\textbackslash end\{beispiele\}}
					\begin{beispiele}[zur Vorlesung] \lipsum[3] \end{beispiele}
				\end{minipage}
			}
			\par
			\fbox{
				\begin{minipage}[t]{.9\textwidth}
					\texttt{\textbackslash begin\{beweis\}[des Satzes 42]  \\ \hspace*{2em}
						\textbackslash lipsum[3]  \\ 
					\textbackslash end\{beweis\}}
					\begin{beweis}[des Satzes 42] \lipsum[3] \end{beweis}
				\end{minipage}
			}
		\section{Die Umgebungen \texttt{quotex} und \texttt{quotationx}}
			\label{quotex_quotationx}
			\newcommand{\pangramm}[1][]{%
				Asynchrone Bas\ifthenelse{\equal{#1}{}}{s}{s:}klänge vom Jazzquintett sind nix für spießige Löwen.\\
				abcdefghijklmnopqr\ifthenelse{\equal{#1}{}}{}{s:}s{}tuvwxyzäüöß \hspace*{2em} 0123456789\\
				ABCDEFGHIJKLMNOPQRSTUVWXYZÄÜÖ\\
				Zażółć gęślą jaźń%
			}
			
			Die Umgebungen haben zwei Parameter, wobei beide optional sind
			\begin{itemize}
				\item Der erste Parameter kann beispielsweise der Urheber eines Zitats oder die Angabe eine Quelle sein, \zb \texttt{Donald Knuth}.
				\item Der zweite Parameter kann einer der Folgenden sein:
					\begin{itemize}
						\item \texttt{tt}: {\tt Gibt den Text mit Typewriter-Schrift aus.}
						\item \texttt{hand}: {\oeschfamily Gibt den Text in Handschrift aus.}
						\item \texttt{fraktur}: {\frakfamily Gibt den Text mit Frakturschrift aus:.} Das Ende-s ("`Rund-s"')wird als \verb|s:| notiert.
					\end{itemize}
					Jeder andere Parameter führt zur Verwendung der Normalschrift. Ist dies gewünscht, sollte man den Parameter allerdings einfach weglassen. 
					
					\newpage
					Als Beispiel ist das \href{https://de.wikipedia.org/wiki/Pangramm\#Liste_deutscher_Pangramme}{Pangramm} 
					\begin{quote}
						\pangramm
					\end{quote}
					gegeben.\\
					\emph{Hinweis:} Im Frakturschrift-Beispiel wurde das {\frakfamily s} zu {\frakfamily s:} angepasst, sofern erforderlich.
			\end{itemize}
			
			\begin{redbox}
				Die polnischen Sonderzeichen \k a und \k e stehen weder in der \texttt{hand}- noch der \texttt{fraktur}-Umgebung zur Verfügung. Sie werden dort jedoch nachgebildet (vgl. folgende Beispiele). Wer die Zeichen durch \texttt{a} bzw. \texttt{e} ohne Ogonek dargestellt haben möchte, muss \verb|\let\k\relax| nach \texttt{\textbackslash begin\{quotex\}} schreiben. 
			\end{redbox}
			
			\begin{redbox}
				Das Paket \texttt{oesch.sty} steht auf manchen Systemen nicht zur Verfügung. In diesem Fall wird die {\oeschfamily Schreibschrift} durch eine einfache \emph{Kursivschrift} ersetzt. Es hift, wenn man die \href{https://www.ctan.org/tex-archive/fonts/oesch?lang=de}{Paketdatei von CTAN herunterlädt} und in den Ordner der Hauptdatei extrahiert. Die \texttt{Beispiel}- und \texttt{Readme}-Dateien können gelöscht werden.
			\end{redbox}
			
			\fbox{
				\begin{minipage}[t]{.9\textwidth}
					\texttt{\textbackslash begin\{quotex\}[Wikipedia][\ellipse] \\ \hspace*{2em}
							\textbackslash pangramm \\ 
					\textbackslash end\{quotex\}}
					\begin{quotex}[Wikipedia (ohne)] 
						\pangramm
					\end{quotex}
					\begin{quotex}[Wikipedia (tt)][tt] 
							\pangramm
					\end{quotex}
					\begin{quotex}[Wikipedia (hand)][hand] 
						\pangramm
					\end{quotex}
					\begin{quotex}[Wikipedia (fraktur)][fraktur]
						\pangramm[!]
					\end{quotex}
					\begin{quotex}[Wikipedia (hand mit \texttt{\textbackslash let\textbackslash k\textbackslash relax})][hand]
						\let\k\relax \pangramm
					\end{quotex}
				\end{minipage}	
			}
		\section{Abstände}
			\label{abstaende beweis beispiel}
			Wenn die \texttt{beweis}- und den \texttt{beispiel(e)}-Umgebungen mit Aufzählungen beginnen bzw. enden, wird sowohl oben als auch unten ein zu großer Abstand eingefügt:
			
			\fbox{
				\begin{minipage}[t]{.9\textwidth}
					\texttt{\textbackslash begin\{beweis\}[für P \$\textbackslash not=\$ NP] \\ \hspace*{2em}
						\textbackslash begin\{itemize\} \textbackslash item Text \textbackslash item Text \textbackslash end\{itemize\} \\ 
						\textbackslash end\{beweis\}
					}
					\begin{beweis}[für P $\not=$ NP]
						\begin{itemize} \item Text \item Text \end{itemize}
					\end{beweis}
				\end{minipage}
			}
			
			\vspace*{.5em}
			Der zweite -- ebenfalls optionale -- Parameter beeinflusst die Abstände oben und unten:
			\begin{itemize}
				\item \texttt o sorgt dafür, dass \texttt{itemize}- und ähnliche Umgebungen den richtigen Abstand nach oben haben;
				\item \texttt u sorgt dafür, dass das $\Box$-Symbol auf richtiger Höhe ist (betrifft ebenfalls \texttt{itemize}, jedoch natürlich nur Beweise -- der Befehl existiert trotzdem auch für die \texttt{beispiel}-Umgebung);
				\item \texttt b schaltet beides ein.
			\end{itemize}
			
			\fbox{
				\begin{minipage}[t]{.9\textwidth}
					\texttt{\textbackslash begin\{beweis\}[für P \$\textbackslash not=\$ NP][b] \\ \hspace*{2em}
						\textbackslash begin\{itemize\} \textbackslash item Text \textbackslash item Text \textbackslash end\{itemize\} \\ 
						\textbackslash end\{beweis\}
					}
					\begin{beweis}[für P $\not=$ NP][b]
						\begin{itemize} \item Text \item Text \end{itemize}
					\end{beweis}
				\end{minipage}
			}
			
	\chapter{Farbige Boxen}
		\section{Standardboxen}
			Es gibt blaue, grüne und gelbe Boxen. Sie sind alle gleich aufgebaut. Die Umgebungen heißen \verb|blueboxIdx|, \verb|yellowboxIdx| und \verb|greenboxIdx|.
			\begin{itemize}
				\item Der erste Parameter ist ein optionales Label, auf den mit \verb|\ref| oder \verb|\vref| zugegriffen werden kann.
				\item Der zweite Parameter ist der Titel der Box.
			\end{itemize}
			Auf einen Zähler kann mit \verb|\boxnummer| zugegriffen werden. \par
			\fbox{
				\begin{minipage}[t]{.9\textwidth}
					\texttt{\textbackslash begin\{blueboxIdx\}[Def einer blauen Box]\{Definition \textbackslash boxnummer: Eine Box\} \\ \hspace*{2em}
						\textbackslash lipsum[3]\\ 
						\textbackslash end\{blueboxIdx\}
					}
					\begin{blueboxIdx}[Def einer blauen Box]{Definition \boxnummer: Eine Box}
						\lipsum[3]
					\end{blueboxIdx}\par
					Die Informationen sind in der Box \vref{Def einer blauen Box} (= \texttt{\textbackslash vref\{Def einer blauen Box\}}) zu finden.
				\end{minipage} 
			}
		\newpage
		\section{Rote Boxen}
			Rote Boxen (\verb|redbox|) haben den Standardtitel \texttt{Achtung!} und keinen Index. \par
			\fbox{
				\begin{minipage}[t]{.9\textwidth}
					\texttt{\textbackslash begin\{redbox\} \\ \hspace*{2em}
						Dieser Satz ist wichtig.\\ 
						\textbackslash end\{redbox\}
					}
					\begin{redbox}
						Dieser Satz ist wichtig.
					\end{redbox}
				\end{minipage} 
			}		
			
	\chapter{Nützlichkeiten}
		\section{Codeschnipsel}
			Dieses Kapitel listet ein paar Codeschnipsel auf, die beim täglichen Gebrauch von \LaTeX\ helfen können.
			\begin{description}
				\item[Bedeutung eines Makros:]
					\verb|\meaning\<makroname>|. Beispielsweise liefert\\\verb|\meaning\makeatletter| Folgendes: \texttt{\meaning\makeatletter}
				\item[Links setzen:]
					\verb|\href{URL}{text}|: Beispielsweise ein Link \href{https://de.wikipedia.org/wiki/LaTeX}{zu Wikipedia}.
			\end{description}
			
		\section{Makroprogrammierung in ${\mathrm {T\!_{\displaystyle E}\!X}}$}
			Der folgende Code implementiert ein einfaches Array:
			\begin{verbatim}
\newcounter{foo}

\makeatletter
    \def\pisarzy@i 	{Krzysztof Kamil Baczyński}
    \def\pisarzy@ii	{Stanisław Barańczak}
    \def\pisarzy@iii{Anna Brzezińska}
    \def\pisarzy@iv {Ewa Lipska}
\makeatother

\def\pisarzy[#1]{%                            <-- Anmerkung
    \setcounter{foo}{#1}%
    \item %
    \makeatletter %
    \csname pisarzy@\roman{foo}\endcsname%
    \makeatother %
}
			\end{verbatim}
			\newcounter{foo}
			
			\makeatletter
				\def\pisarzy@i 	{Krzysztof Kamil Baczyński}
				\def\pisarzy@ii	{Stanisław Barańczak}
				\def\pisarzy@iii{Anna Brzezińska}
				\def\pisarzy@iv {Ewa Lipska}
			\makeatother
			
			\def\pisarzy[#1]{%
				\setcounter{foo}{#1}%
				\item %
				\makeatletter %
				\csname pisarzy@\roman{foo}\endcsname%
				\makeatother %
			}
			
			\newpage
			So greift man auf die Felder zu: 
			
			\begin{verbatim}
\begin{itemize}
    \pisarzy[1]
    \pisarzy[2]
    \pisarzy[3]
    \pisarzy[4]
\end{itemize}
			\end{verbatim}
			
			\begin{itemize}
				\pisarzy[1]
				\pisarzy[2]
				\pisarzy[3]
				\pisarzy[4]
			\end{itemize}
			
			\emph{Anmerkung:} Man könnte hier auch \verb|\def\pisarzy--#1--]| schreiben, dann greift man auf die Felder mit \verb|\pisarzy --42--]| zu.
			
			\emph{Basierend auf} dem \href{http://pgfplots.sourceforge.net/TeX-programming-notes.pdf}{Skript von C. Feuersänger}.
	% ------------------------------------------------------------------------------------------------------------------

%	\appendix		
%	\chapter{Änderungen/Changelog}
%		\begin{longtable}{|c|p{10cm}|}
%			\hline
%			\bf Version 				& \bf Änderungen 											\\
%			\hline
%			5.15						& 	\begin{itemize}
%												\item[+] Neues Paket \texttt{todonotes} hinzu.		
%											\end{itemize} 											\\
%			\hline
%		\end{longtable}
\end{document}
