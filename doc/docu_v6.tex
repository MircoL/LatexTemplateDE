\def\dokumentTyp{Skript}

\def\hauptsprache{ngerman}
\def\andereSprachen{}
\input{../mircol-v6}

\pgfqkeys{/VorlageVersion5}{
	all/Autoren						= {Mirco Lukas},
	all/Titel						= {Dokumentation zur \LaTeX-Vorlage},
	all/Untertitel					= {Eine \LaTeX-Vorlage mit vielen vordefinierten Befehlen\par Version~5.15},
	all/VersionPraefix				= {},
	all/Version						= {false},
	all/Icon/Breite					= {1},
	all/Icon/URL					= {http://www.el-celta.de/latex2.gif},
	all/Index/Boxen/blau/zeigen		= {false},
	all/Index/Boxen/blau/titel		= {Liste der Sätze und Definitionen},
	all/Index/Boxen/gelb/zeigen		= {false},
	all/Index/Boxen/gelb/titel		= {Liste der Beispiele},
	all/Index/Boxen/gruen/zeigen	= {false},
	all/Index/Boxen/gruen/titel		= {Liste der ...},
	all/Index/Begriffe/zeigen		= {true},
	all/Index/Begriffe/titel		= {B\ \ Index},
	all/Index/Literatur/zeigen		= {false},
	all/Index/Literatur/titel		= {Literaturverzeichnis},
}


\usepackage{pdfpages}
\newcommand{\incpdf}[1]{\includepdf[pages=-,frame= true,delta=3mm 3mm]{#1}}
\newcommand{\doubleplus}{double\textsuperscript{+}}
\newcommand{\pangramm}[1][]{%
	Asynchrone Bas\ifthenelse{\equal{#1}{}}{s}{s:}klänge vom Jazzquintett sind nix für spießige Löwen.\\
	abcdefghijklmnopqr\ifthenelse{\equal{#1}{}}{}{s:}s{}tuvwxyzäüöß \hspace*{2em} 0123456789\\
	ABCDEFGHIJKLMNOPQRSTUVWXYZÄÜÖ\\
	Zażółć gęślą jaźń%
}

\begin{document}

	\chapter*{Weblink zum Download der Vorlage}
		Updates der Vorlage und dieser Dokumentation können auf \name{GitHub} heruntergeladen werden:
		\begin{center}
			\url{https://github.com/MircoL/LatexTemplateDE}
		\end{center}
	

	\chapter{Grundeinstellungen für Dokumente}
		\section{Der \texttt{\textbackslash dokumentTyp}}
			\index{dokumentTyp@\texttt{dokumentTyp}}
			Es gibt aktuell fünf DokumentTypen. Diese führen zu verschiedenen Dokumentausgaben.
			\begin{tabbing}
				mmmmmmmmmmmmmmm				\= mmmmmmmmmmmmmmmm					 \kill
				\textbf{Der \texttt{dokumentTyp}\ellipse}
												\> \textbf{\ellipse erzeugt die \texttt{documentclass}} \\
				\texttt{Skript}					\> \texttt{scrbook}\\
				\texttt{Mitschrift}				\> \texttt{scrartcl} \\
				\texttt{Buch}					\> \texttt{scrbook} \\
				\texttt{Praesentation}			\> \texttt{beamer} \\
				\texttt{Thesis}					\> \texttt{srcbook} \\
				\emph{(ein anderer)}		\> \em (Fehlermeldung: Ungueltiger Dokumenttyp.)
			\end{tabbing}
		Die einzelnen Dokumente, die erzeugt werden, sehen wie folgt aus:

		\newpage
		\begin{center}
			\vspace*{20em}
			\subsection{Beispiel zu \texttt{Skript}}
		\end{center}
		\incpdf{muster-skript.pdf}
		\newpage

		\begin{center}
			\vspace*{20em}
			\subsection{Beispiel zu \texttt{Mitschrift}}
		\end{center}
		\incpdf{muster-mitschrift.pdf}
		\newpage

		\begin{center}
			\vspace*{20em}
			\subsection{Beispiel zu \texttt{Buch}}
		\end{center}
		\incpdf{muster-buch.pdf}
		\newpage

		\begin{center}
			\vspace*{20em}
			\subsection{Beispiel zu \texttt{Praesentation}}
		\end{center}
		\incpdf{muster-praesentation.pdf}
		\newpage


		\begin{center}
			\vspace*{20em}
			\subsection{Beispiel zu \texttt{Thesis}}
		\end{center}
		\incpdf{muster-thesis.pdf}
		\label{thesis-bsp}
		\newpage



	\section{Dokumenteinstellungen}
		\subsection{Einstellungen vor dem Einbinden der Datei \texttt{mircol-v6.tex}}
			Die folgenden Einstellungen beeinflussen die Pakete \texttt{babel} und \texttt{varioref}:
			\begin{tabbing}
				mmmmmmmmmmm 			\= m \kill
				\textbf{Befehl}			\> \textbf{Erklärung} \\
				\verb|\hauptsprache|	\> Die Hauptsprache des Dokuments, \zb \verb|ngerman|, \verb|english|, \\
										\> \verb|polish|, \verb|russian|  \\
				\verb|\andereSprachen|	\> eine Liste weiterer Sprachen, die im Dokument vorkommen \\
										\> (durch Kommata getrennt)
			\end{tabbing}
			Die folgenden Einstellungen beeinflussen den Zeilenabstand im Dokument:
			\begin{tabbing}
				mmmmmmmmmmm 			\= mmmmmmmmmmmmmmmmmmm \= \kill
				\textbf{Befehl}			\> \textbf{Erklärung}			\> \textbf{Wertebereich} \\
				\verb|\zeilenabstand|	\> Der gewünschte Zeilenabstand \> \texttt{singlespacing}, \texttt{onehalfspacing} \\
										\>								\> oder \texttt{doublespacing}
			\end{tabbing}

		\subsection{Arrays}
			Es werden folgende Arrays definiert. Die einzelnen Felder werden \emph{ohne Leerzeichen} mit "`\&"' voneinander getrennt: \verb|all/Autoren = {Alice&Bob}|.
			\begin{tabbing}
				mmmmmmmmmmmmmmm				\= mmmmmmmmmmmmmmmmmmmmm						\=	mmmmmmmmmm					\kill
				\textbf{Name des Arrays}	\> \textbf{Inhalt}								\> \textbf{Dokumentklassen}		\\
				 all/Autoren				\> Die Autoren der Dokumente \emph{(Anm. a)}	\> alle							\\
				 Thesis/EingereichtBei		\> Prof., bei dem eingereicht wird				\> \texttt{Thesis} 				\\
				 Thesis/BetreuungDurch		\> Die Betreuer der Thesis						\> \texttt{Thesis}
			\end{tabbing}

			\paragraph*{Anmerkungen:}
				\begin{enumeratealpha}
					\item Wird kein Autor angegben, so wird beim Dokumenttyp \texttt{Thesis} eine Fehlermeldung geworfen und keine PDF ausgegeben.
				\end{enumeratealpha}

		\subsection{Pgfkeys}
			Im Folgenden werden die existierenden Schlüssel-Werte-Paare gelistet. Die Dokumentklassen sind aus dem Schlüssel direkt ersichtlich. Dabei bedeuten:
			\begin{description}
				\item[string:]
						Ein beliebiger Text. Zeilenumbrüche sind mit \verb|\par| zu definieren.
				\item[boolean:]
						Genau eines der Werte \texttt{true} oder \texttt{false}.
				\item[integer:]
						Eine positive, ganze Zahl ($n \in \mathbb N^+ = \{1, \, 2, \, \ldots\}$)
				\item[\doubleplus:]
						Eine positive Dezimalzahl ($n \in \mathbb Q^+$) mit Dezimalpunkt, \zb "`2.3"'
				\item[length:]
						Eine Zahl plus Einheit, \zb "`12em"'; "`42pt"'
				\item[url:]
						Eine Internetadresse.
				\item[{\texttt{ \{\ldots\}}}:]
						Genau einer der genannten Werte.
			\end{description}
			\enlargethispage*{2em}
			\begin{tabbing}
				mmmmmmmmmmmmmmmmmm				\= mmmmmmmmmmmmmmmmmmmmmm 								\=mmm\=\kill
				\textbf{Schlüssel}				\> \textbf{Bedeutung}									\> \textbf{Wertebereich}	\\
				all/Titel	 					\> Titel der Arbeit	\emph{(Anm. a)}						\> string					\\
				all/Untertitel 					\> Untertitel											\> string					\\
				all/VersionPraefix 				\> Präfix der Versionsangabe							\> string					\\
				all/Version 					\> Anzeigen der Version									\> boolean 					\\
				all/Icon/Breite 				\> Breite des Titelbildes \emph{(Anm. b)}				\> \doubleplus				\\
				all/Icon/URL					\> URL des Titelbildes									\> url						\\
				all/Index/Boxen/blau/zeigen 	\> Zeige den Index aller blauen Boxen					\> boolean 					\\
				all/Index/Boxen/blau/titel 		\> Titel des Index' aller blauen Boxen					\> string					\\
				all/Index/Boxen/gelb/zeigen 	\> Zeige den Index aller gelben Boxen					\> boolean 					\\
				all/Index/Boxen/gelb/titel 		\> Titel des Index' aller gelben Boxen					\> string					\\
				all/Index/Boxen/gruen/zeigen 	\> Zeige den Index aller grünen Boxen					\> boolean 					\\
				all/Index/Boxen/gruen/titel 	\> Titel des Index' aller grünen Boxen					\> string					\\
				all/Index/Begriffe/zeigen 		\> Zeige den Index der Begriffe							\> boolean					\\
				all/Index/Begriffe/titel 		\> Titel des Index' der Begriffe						\> string					\\
				all/Index/Literatur/zeigen		\> Zeige das Literaturverzeichnis \emph{(Anm. c)}		\> boolean					\\
				all/Index/Literatur/titel		\> Titel des Literaturverzeichnisses	\> string									\\
				\\
				Skript/AnmerkungenTitelseite 	\> Eine Anmerkung, die auf der Titelseite\> string									\\
												\> erscheint																		\\
				\\
				Mitschriften/Vorlesungsname 	\> Name der Vorlesung 					\> string									\\
				Mitschriften/Typ 				\> \zb \emph{Übung, Vorlesung, Praktikum}
																						\> string									\\
				Mitschriften/LfdNr 				\> \zb die jeweilige Übungsnummer		\> string (!)								\\
				Mitschriften/Gruppe 			\> \zb die Übungsgruppe					\> string									\\
				Mitschriften/Headerhoehe		\> Höhe des Headers (falls Text hineinragt)
																						\> length									\\
				\\
				Buecher/Widmung					\> Eine Widmung							\> string									\\
				Buecher/Modus					\> Aufteilung in rechte und linke Seite (\texttt{rl})
																						\> \{\texttt{rl}, \texttt{e}\} 				\\
												\> oder jede Seite einzeln und zentriert (\texttt{e})								\\
				\\
				Praesentationen/TitelKurz		\> Ein Kurztitel (für die Fußzeile)		\> string			\\
				Praesentationen/Institut/lang	\> Das Institut (für die Titelseite)	\> string			\\
				Praesentationen/Institut/kurz	\> Abkürzung des Instituts 				\> string			\\
												\> (für die Fußzeile)					\> \> ...			\\
				Thesis/AkademischerGrad/kurz	\> \zb \texttt{Bachelor}				\> string			\\
				Thesis/AkademischerGrad/lang	\> \zb \texttt{Bachelor of Science}		\> string			\\
				Thesis/Fach/Nominativ			\> Das Fach im Nominativ				\> string			\\
				Thesis/Fach/Genitiv				\> Das Fach im Genitiv (einschl. Artikel) \> string			\\
				Thesis/UniName/lang				\> Der Name der Uni (Langform)			\> string			\\
				Thesis/UniName/kurz				\> Der Name der Uni (Kurzform)			\> string			\\
				Thesis/Abgabedatum				\> Das Abgabedatum der Thesis			\> string			\\
				Thesis/Versicherung/zeigen		\> Zeige die Versicherung der Eigenleistung					\\
												\> \emph{(Anm. d)}						\> boolean			\\
				Thesis/Versicherung/text		\> Text der Versicherung 				\> string			\\
				Thesis/Versicherung/ort			\> \idr der Sitz der Uni 				\> string
			\end{tabbing}
			\paragraph*{Anmerkungen:}
			\begin{enumeratealpha}
				\item
					Wird kein Titel angegben, so wird beim Dokumenttyp \texttt{Thesis} eine Fehlermeldung geworfen und keine PDF ausgegeben.
					Als Titelbild wird die Datei \texttt{titel.jpg} verwendet.
					Ist diese Datei im Arbeitsverzeichnis nicht vorhanden, bleiben die dazugehörigen Befehle ohne Wirkung.
				\item
					Als Literaturliste wird die Datei \texttt{quellen.bib} verwendet.
					Ist diese Datei im Arbeitsverzeichnis nicht vorhanden, so wird keine Literaturliste erzeugt.
				\item
					Wird die Anzeige der Versicherung eingeschaltet, aber kein Text angegeben, so erscheint ein Standardtext (Siehe Muster im Kap. \vref{thesis-bsp})

			\end{enumeratealpha}


	% ------------------------------------------------------------------------------------------------------------------


	\chapter{Eigene Befehle}
		\section{Allgemeine Befehle}
			Ein Stern gibt an, dass der jeweilige Parameter optional ist.
			\begin{tabbing}
				mmmmmmmmmm			\= mmmmmmmmmmmmmmmm							\= mmmmmmmmmm			\= \kill
				\textbf{Befehl}		\> \textbf{Beschreibung/Anmerkung}			\> \textbf{Ergebnis}	\> \textbf{Parameter}				\\
				\verb|\andd|		\> Ein Synonym für \verb|\wedge|			\> $\andd$ 				\> ---								\\
				\verb|\aufgabe|		\> Nützlich für Übungsblätter				\> Aufgabe \#1 			\> \#1: Nr. der Aufgabe 			\\
				\verb|\bigandd|		\> Ein Synonym für \verb|\bigwedge|			\> $\bigandd$ 			\> --- 								\\
				\verb|\bigorr|		\> Ein Synonym für \verb|\bigvee|			\> $\bigorr$ 			\> --- 								\\
				\verb|\blitz|		\> Ein Synonym für							\> \lightning\xspace 	\> ---								\\
									\> \verb|\lightning\xspace|																				\\
				\verb|\colorref(i)|	\> Farbige Referenzzeichen					\> $i=1:$ \colorref(1)	\> $i \in \{1, \, \ldots, \, 4\}$	\\
									\>(\zb für Beweise)							\> $i=2:$ \colorref(2)										\\
									\>											\> $i=3:$ \colorref(3)										\\
									\>											\> $i=4:$ \colorref(4)										\\
				\verb|\ellipse|		\> Eine Ellipse	 (\href{https://de.wikipedia.org/wiki/Auslassungspunkte}{Wikipedia})
									\> \ellipse				\> ---							\\
				\verb|\fallunterscheidung|\\
									\> Für Fallunterscheidungen					\> vgl. Anm. a)			\> \#1*: Ein Präfix.				\\
									\>											\>						\> \#2: Der obere Fall				\\
									\>											\>						\> \#3: Der untere Fall				\\
				\verb|\fuer|		\> Für Fallunterscheidungen					\> vgl. Anm. a)			\> ---								\\
				\verb|\length|		\> Kurz für $|$Text$|$						\> \length{Text}		\> \#1: ein Text					\\
				\verb|\name|		\> Ein Text in Kapitälchen					\> \name{Der Autor}		\> \#1: ein Name					\\
				\verb|\neuerbegriff|\> Einführung eines Begriffs				\> Siehe Abschn. \ref{begriffsdefinitionen}					\\
				\verb|\neuerbegriffIdx|
									\>Einführung eines Begriffs					\> Siehe Abschn. \ref{begriffsdefinitionen}					\\
				\verb|\orr|			\> Ein Synonym für \verb|\vee|				\> $\orr$ 				\> ---								\\
				\verb|\sonst|		\> Für Fallunterscheidungen					\> vgl. Anm. a)			\> ---								\\
				\verb|\teilaufgabe|	\> Nützlich für Übungsblätter				\> TA \verb|\aufgabe|.\#1
																										\> \#1: Nr. der Teilaufgabe		 	\\
				\verb|\textmarker|	\> Ein Textmarker							\> \textmarker{green}{Text}
																										\> \#1: Farbe des Markers			\\
									\>											\>						\> \#2: Der zu mark. Text 			\\
				\verb|\qed|			\> Eine Box am Ende der Zeile				\> vgl. Anm. b)			\> ---								\\
				\verb|\wiki|		\> Ein Link zu Wikipedia					\>\wiki{https://de.wikipedia.org/wiki/LaTeX}
																										\> \#1: Eine Wiki-URL				\\
									\>  (hier \texttt{https://de.wikipedia.org/wiki/LaTeX})
			\end{tabbing}
			\newpage
			\textbf{Anmerkungen:}
			\begin{enumeratealpha}
				\item \verb|\fallunterscheidung[\operatorname{abs}(x) =]|\\
					  \verb|                              {x \fuer x \geq 0,}{-x \sonst.}| \\
						\frame{
							\begin{minipage}[t]{.9\textwidth}
								\bigskip
								\ \fallunterscheidung[\operatorname{abs}(x) =]{x \fuer x \geq 0,}{-x \sonst.}
								\bigskip
							\end{minipage}
						}
				\item \verb|Dieser Satz ist falsch. \qed| \\
					\frame{
						\begin{minipage}[t]{.9\textwidth}
							\bigskip
							\ Dieser Satz ist falsch. \qed
							\bigskip
						\end{minipage}
					}

			\end{enumeratealpha}
		\section{Umdefinieren vorhandener Befehle}
			Die folgenden \TeX-/\LaTeX-Befehle erhielten eine neue Bedeutung.
			\begin{tabbing}
				mmmmmm				\= mmmmmmmmmmm				\= \kill
				\textbf{Befehl}		\> \textbf{Neue Bedeutung}	\> \textbf{Bemerkung}														\\
				\verb|\appendix|	\> \verb|\appendix|			\> Stellt nun sicher, dass der \LaTeX-Befehl 								\\
									\> 							\> \verb|\appendix| nur einmal aufgerufen wird.								\\
				\verb|\chapter|		\> \verb|\section|			\> nur in \texttt{Mitschrift}en, da dort keine \texttt{chapter} existieren.	\\
				\verb|\epsilon|		\> $\epsilon$				\> $\epsilonOld$ = \verb|\epsilonOld|										\\
				\verb|\phi|			\> $\phi$					\> $\phiOld$ = \verb|\phiOld|
			\end{tabbing}
		\section{Deaktivierung vorhandener Befehle}
			Der Befehl \verb|\let<\befehl>\relax| deaktiviert ein Kommando. So kann man etwa den Befehl \verb|\qed| mit \verb|\let\qed\relax| abschalten:
			\begin{verbatim}
Lorem ipsum dolor sit amet, consectetuer adipiscing elit. \qed   \\
\let\qed\relax
Aenean massa. Cum sociis natoque penatibus et magnis.     \qed
			\end{verbatim}

			\fbox{
				\begin{minipage}[t]{.9\textwidth}
					\bigskip
				\begingroup
					Lorem ipsum dolor sit amet, consectetuer adipiscing elit. \qed   \\
					\let\qed\relax
					Aenean massa. Cum sociis natoque penatibus et magnis.     \qed
				\endgroup
					\bigskip
			\end{minipage}
			}
			\vspace*{1em}\\
			  Das zweite \texttt{qed}-Zeichen wird durch \verb|\relax| unterdrückt.

		\newpage
		\section{Gängige Abkürzungen}
			Die folgenden Befehle erleichtern die Verwendung von Abkürzungen. Zwischen die einzelnen Bestandteile sollte ein Achtelgeviert-Zwischenraum eingefügt wird. Das übliche Leerzeichen ist \idr ein Halbgeviert groß (Mehr Infos in der \wiki{https://de.wikipedia.org/wiki/Geviert_(Typografie)}).
			\begin{tabbing}
				mmmmmmm			\= mmmmmmmmmmmmmm						\= \kill
				Vergleiche		\>\texttt{o.$\sim$B.$\sim$d.$\sim$A.}:	\> Es sei o.~B.~d.~A. $x>0$.\\
				und				\>\verb|\obda|=\verb|o.\,b.\,d.\,a.|:	\> Es sei \obda $x>0$.
			\end{tabbing}
			Alle Befehle gibt es auch mit Großbuchstaben (\verb|\zb| -- \verb|\Zb|) am Anfang, etwa für Satzanfänge.
			\begin{tabbing}
				mmmmmmmmm						\= mmmmmmmmmmmmmmmmmmmmm		\=\kill
				\textbf{Befehl}					\> \textbf{Bedeutung}			\> \textbf{Bemerkung}								\\
				\verb|\dhe|, \verb|\Dhe|		\> das heißt 					\> \verb|\dh| ist schon für \dh\ belegt.			\\
				\verb|\idr|, \verb|\Idr|		\> in der Regel																		\\
				\verb|\obda|, \verb|\Obda|		\> ohne Beschränkung der Allgemeinheit 												\\
				\verb|\sio|, \verb|\Sio|		\> siehe oben 					\> \verb|\so| ist im Paket \texttt{soul} für 		\\
				\verb|\siu|, \verb|\Siu|		\> siehe unten					\> \so{gesperrt} reserviert.						\\
				\verb|\ua|, \verb|\Ua|			\> unter anderem																	\\
				\verb|\va|, \verb|\Va|			\> vor allem																		\\
				\verb|\zb|, \verb|\Zb|			\> zum Beispiel																		\\
				\verb|\zt|,	\verb|\Zt|			\> zum Teil
			\end{tabbing}
		\section{Begriffsdefinitionen}
			\label{begriffsdefinitionen}
			Der Befehl \verb|\neuerbegriff{}| hebt einen Begriff hervor, indem er rot umrahmt und zusätzlich grau hinterlegt wird:
			\begin{verbatim}
				\begin{quotex}[\wiki{https://de.wikipedia.org/wiki/Betriebssystem}]
				      Ein \neuerbegriff{Betriebssystem} ist eine Zusammenstellung
				      von Computerprogrammen\ellipse
				\end{quotex}
			\end{verbatim}

			\fbox{
				\begin{minipage}[t]{\textwidth}
					\bigskip
					\begin{quotex}[\wiki{https://de.wikipedia.org/wiki/Betriebssystem}]
						Ein \neuerbegriff{Betriebssystem} ist eine Zusammenstellung
						von Computerprogrammen\ellipse
					\end{quotex}
					\bigskip
				\end{minipage}
			}

			\newpage
			Der Befehl \verb|\neuerbegriffIdx{}| markiert den durch den Parameter gegebenen Text analog zu \verb|\neuerbegriff{}|, fügt ihn aber zusätzlich in das Stichwortverzeichnis ein. Er funktioniert wie folgt:
			\begin{itemize}
				\item
					Gibt man nur einen Parameter an, so erscheint der Text anstelle des Befehls und der gleiche Text steht im Index: \\
					\begin{verbatim}
						\begin{quotex}[\wiki{https://de.wikipedia.org/wiki/Betriebssystem}]
						     Ein \neuerbegriffIdx{Betriebssystem} ist eine Zusammenstellung
						     von Computerprogrammen, die die Systemressourcen eines
						     Computers wie Arbeitsspeicher, Festplatten, Ein- und
						     Ausgabegeräte verwaltet und diese Anwendungsprogrammen
						     zur Verfügung stellt.
						\end{quotex}
					\end{verbatim}

					\fbox{
						\begin{minipage}[t]{\textwidth}
							\bigskip
							\begin{quotex}[\wiki{https://de.wikipedia.org/wiki/Betriebssystem}]
								Ein \neuerbegriffIdx{Betriebssystem} ist eine Zusammenstellung
								von Computerprogrammen, die die Systemressourcen eines
								Computers wie Arbeitsspeicher, Festplatten, Ein- und
								Ausgabegeräte verwaltet und diese Anwendungsprogrammen
								zur Verfügung stellt.
							\end{quotex}
							\bigskip
						\end{minipage}
					}

					Im Index wird \texttt{Betriebssystem} stehen.

				\item
					Der optionale Parameter bietet die Möglichkeit, einen abweichenden Text in den Index zu schreiben. Dies ist nützlich für Index-Befehle oder falls der Begriff flektiert ist:
					\begin{verbatim}
						\begin{quotex}[\wiki{https://de.wikipedia.org/wiki/Betriebssystem}]
						     Betriebssysteme bestehen in der Regel aus einem Kernel [\ellipse].
						     Zu diesen Aufgaben gehört unter anderem das Laden von
						     \neuerbegriffIdx[Gerätetreiber, Einführung]{Gerätetreibern}.
						\end{quotex}
					\end{verbatim}

					\fbox{
						\begin{minipage}[t]{.9\textwidth}
							\bigskip
								\begin{quotex}[\wiki{https://de.wikipedia.org/wiki/Betriebssystem}]
									Betriebssysteme bestehen in der Regel aus einem Kernel [\ellipse].
									Zu diesen Aufgaben gehört unter anderem das Laden von
									\neuerbegriffIdx[Gerätetreiber, Einführung]{Gerätetreibern}.
								\end{quotex}
						\end{minipage}
					}

					Im Index wird \texttt{Gerätetreiber, Einführung} stehen.
			\end{itemize}


	% ------------------------------------------------------------------------------------------------------------------


	\chapter{Eigene Umgebungen}
		\section{Übersicht}
			Die Umgebungen werden mit \verb|\begin{name} ... \end{name}| umschlossen. Ein Stern gibt an, dass der Parameter optional ist.
			\begin{tabbing}
				mmmmmmmmmm				\= mmmmmmmmmmmmmmmmmmm							\= mmmmmmmmmmm																					\kill
				\textbf{Befehl}			\> \textbf{Beschreibung}						\> \textbf{Parameter} 																			\\
				\verb|beispiel|			\> Eine \textbf{Beispiel}-Umgebung 				\> \#1${}^*$: Text hinter \emph{Beispiel} 														\\
										\>												\> \#2${}^*$: $x \in \{o, \, u, \, b\}$ (Abschn. \ref{abstaende beweis beispiel}) 				\\
				\verb|beispiele|		\> Eine \textbf{Beispiele}-Umgebung 			\> \#1${}^*$: Text hinter \emph{Beispiele} 														\\
										\>												\> \#2${}^*$: $x \in \{o, \, u, \, b\}$ (Abschn. \ref{abstaende beweis beispiel}) 				\\
				\verb|beweis|			\> Eine Umgebung, die mit 						\> \#1${}^*$: Text hinter \emph{Beweis}															\\
										\> dem \emph{qed}-Symbol ($\Box$) endet			\> \#2${}^*$: $x \in \{o, \, u, \, b\}$ (Abschn. \ref{abstaende beweis beispiel})				\\
				\verb|enumeratealpha|	\> Eine Auflistung mit a), b), \ellipse			\> ---																							\\
				\verb|enumerateAlpha|	\> Eine Auflistung mit A), B), \ellipse 		\> ---																							\\
				\verb|enumerateroman|	\> Eine Auflistung mit i), ii), \ellipse		\> ---																							\\
				\verb|enumerateRoman|	\> Eine Auflistung mit I), II), \ellipse		\> ---																							\\
				\verb|quotex|			\> Eine \texttt{quote}-Umgebung mit Extras
																						\> Siehe Abschnitt \ref{quotex_quotationx}.														\\
				\verb|quotationx|		\>	Analog zu \texttt{quotex}					\> Siehe Abschnitt \ref{quotex_quotationx}.
			\end{tabbing}

			\noindent \textbf{Im Folgenden einige Beispiele.}

			\begin{verbatim}
				\begin{enumeratealpha}
				     \item ein Item
				     \item ein Item
				\end{enumeratealpha}
			\end{verbatim}

			\fbox{
				\begin{minipage}[t]{.9\textwidth}
					\bigskip
					\begin{enumeratealpha}
						\item ein Item
						\item ein Item
					\end{enumeratealpha}
					\bigskip
				\end{minipage}
			}

			\newpage

			\begin{verbatim}
				\begin{beispiel}
				     \lipsum[3]
				\end{beispiel}
			\end{verbatim}

			\fbox{
				\begin{minipage}[t]{.9\textwidth}
					\bigskip
					\begin{beispiel}
						\lipsum[3]
					\end{beispiel}
					\bigskip
				\end{minipage}
			}

			\begin{verbatim}
				\begin{beispiele}[zur Vorlesung]
				     \lipsum[3]
				\end{beispiele}
			\end{verbatim}

			\fbox{
				\begin{minipage}[t]{.9\textwidth}
					\bigskip
					\begin{beispiele}[zur Vorlesung]
						\lipsum[3]
					\end{beispiele}
					\bigskip
				\end{minipage}
			}

			\begin{verbatim}
				\begin{beweis}[des Satzes 42]
				    \lipsum[3]
				\end{beweis}
			\end{verbatim}

			\fbox{
				\begin{minipage}[t]{.9\textwidth}
					\bigskip
					\begin{beweis}[des Satzes 42]
						\lipsum[3]
					\end{beweis}
					\bigskip
				\end{minipage}
			}

		\section{Die Umgebungen \texttt{quotex} und \texttt{quotationx}}
			\label{quotex_quotationx}

			Die Umgebungen \texttt{quotex} und \texttt{quotationx} haben jeweils zwei Parameter, wobei beide optional sind.
			\begin{itemize}
				\item Der erste Parameter kann beispielsweise der Urheber eines Zitats oder die Angabe eine Quelle sein, \zb \texttt{Donald Knuth}.
				\item Der zweite Parameter kann einer der Folgenden sein:
					\begin{itemize}
						\item \texttt{tt}: {\texttt{Gibt den Text mit Typewriter-Schrift aus.}}
						\item \texttt{fraktur}: {\frakfamily Gibt den Text mit Frakturschrift aus:.} Das Ende-s ("`Rund-s"') wird als \verb|s:| notiert.
					\end{itemize}
					Jeder andere Parameter führt zur Verwendung der Normalschrift. Ist dies gewünscht, sollte man den Parameter allerdings einfach weglassen.

					\newpage
					Als Beispiel ist das \href{https://de.wikipedia.org/wiki/Pangramm\#Liste_deutscher_Pangramme}{Pangramm}
					\begin{quote}
						\pangramm
					\end{quote}
					gegeben.\\
					\emph{Hinweis:} Im Frakturschrift-Beispiel wurde das {\frakfamily s} zu {\frakfamily s:} angepasst, sofern erforderlich.
			\end{itemize}

			\begin{redbox}
				Die polnischen Sonderzeichen \k a und \k e stehen in der \texttt{hand}- und der \texttt{fraktur}-Umgebung nicht zur Verfügung. Sie werden dort jedoch nachgebildet (vgl. folgende Beispiele). Wer die Zeichen durch \texttt{a} bzw. \texttt{e} ohne Ogonek dargestellt haben möchte, muss \verb|\let\k\relax| nach \texttt{\textbackslash begin\{quotex\}} schreiben.
			\end{redbox}

			\begin{verbatim}
				\begin{quotex}[Wikipedia (ohne Parameter)]
				    \pangramm
				\end{quotex}
			\end{verbatim}

			\fbox{
				\begin{minipage}[t]{.9\textwidth}
					\bigskip
					\begin{quotex}[Wikipedia (ohne  Parameter)]
						\pangramm
					\end{quotex}
					\bigskip
				\end{minipage}
			}

			\begin{verbatim}
				\begin{quotex}[Wikipedia mit Parameter "`tt"'][tt]
				    \pangramm
				\end{quotex}
			\end{verbatim}

			\fbox{
				\begin{minipage}[t]{.9\textwidth}
					\bigskip
					\begin{quotex}[Wikipedia mit Parameter "`tt"'][tt]
						\pangramm
					\end{quotex}
					\bigskip
				\end{minipage}
			}

			\begin{verbatim}
				\begin{quotex}[Wikipedia mit Parameter "`fraktur"'][fraktur]
				    \pangramm[:]
				\end{quotex}
			\end{verbatim}

			\fbox{
				\begin{minipage}[t]{.9\textwidth}
					\bigskip
					\begin{quotex}[Wikipedia mit Parameter "`fraktur"'][fraktur]
						\pangramm[:]
					\end{quotex}
					\bigskip
				\end{minipage}
			}

			\begin{verbatim}
				\begin{quotex}
				          [Wikipedia mit Parameter "`fraktur"' und
					                \texttt{\textbackslash let%
					                \textbackslash k%
					                \textbackslash relax}
					      ]
				          [fraktur]
				    \let\k\relax
				    \pangramm[:]
				\end{quotex}
			\end{verbatim}

			\fbox{
				\begin{minipage}[t]{.9\textwidth}
					\bigskip
						\begin{quotex}
							[Wikipedia mit Parameter "`fraktur"' und
							\texttt{\textbackslash let%
								\textbackslash k%
								\textbackslash relax}
							]
							[fraktur]
							\let\k\relax
							\pangramm[:]
						\end{quotex}
					\bigskip
				\end{minipage}
			}






		\section{Die Umgebung \texttt{documentationx}}
			Diese Umgebung dient zur Dokumentation von Quellcode.
			Die folgenden Befehle erhalten jeweils
			\begin{itemize}
				\item als ersten Parameter einen eindeutigen\footnote{Es genügt, wenn der Name innerhalb einer Verschachtelung eindeutig ist.} Namen
				\item als zweiten Parameter einen solchen in der Form \verb|key={value}| -- genauso wie die Definitionen am Anfang der Vorlage.
			\end{itemize}
			Es stehen zur Verfügung:

			\begin{itemize}
				\item \verb|\Pkgdef| mit den Parametern \texttt{name} und \texttt{description};
				\item \verb|\Classdef|, \verb|\Methoddef| und \verb|\Vardef| zusätzlich mit den Parametern \texttt{modifier} und \texttt{type}.
			\end{itemize}
			Darüber hinaus gibt es den Befehl \verb|\Parameter|. Der erste (optionale) Parameter entspricht dem Variablentyp, der zweite den Variablennamen.
			Die Farbgebung orientiert sich am Syntax-Highlighting von Eclipse.

			\newpage
			\enlargethispage*{1em}
			\begin{verbatim}
\begin{documentationx}
   \Pkgdef{org.example.java.test}{
     name={org.example.java.test},
     description={
       Dieses Paket ist nur ein Test. Es enhält folgende Klassen: \\
       \Classdef{org.example.java.test.PossibleTestCases}{
         modifier={public},
         type={enum},
         name={PossibleTestCases},
         description={
            Eine Reihe von Testfällen.
         }
       }
       \Classdef{org.example.java.test.Functions}{
         modifier={public},
         type={class},
         name={Functions},
         description={
            Eine Klasse mit vielen tollen Funktionen, \zb:\\
            \Methoddef{org.example.java.test.Functions::getValue0}{
              modifier={public},
              type={void},
              name={setValue(\Parameter[int]{value})},
              description={
                Setzt einen wichtigen Wert.
              }
            }
            \noindent Außerdem gibt es noch diese Variable:\\
            \Vardef{org.example.java.test.Functions::value}{
              modifier={private},
              type={int},
              name={value},
              description={
                Ein Wert.
              }
            }
         }
       }
     }
   }
\end{documentationx}
			\end{verbatim}

	\noindent\makebox[\linewidth]{\rule{.7\paperwidth}{0.4pt}}
				\begin{documentationx}
					\Pkgdef{org.example.java.test}{
						name={org.example.java.test},
						description={
							Dieses Paket ist nur ein Test. Es enhält folgende Klassen: \\
							\Classdef{org.example.java.test.PossibleTestCases}{
								modifier={public},
								type={enum},
								name={PossibleTestCases},
								description={
									Eine Reihe von Testfällen.
								}
							}
							\Classdef{org.example.java.test.Functions}{
								modifier={public},
								type={class},
								name={Functions},
								description={
									Eine Klasse mit vielen tollen Funktionen, \zb:\\
									\Methoddef{org.example.java.test.Functions::getValue0}{
										modifier={public},
										type={void},
										name={setValue(\Parameter[int]{value})},
										description={
											Setzt einen wichtigen Wert.
										}
									}
									\noindent Außerdem gibt es noch diese Variable:\\
									\Vardef{org.example.java.test.Functions::value}{
										modifier={private},
										type={int},
										name={value},
										description={
											Ein Wert.
										}
									}
								}
							}
						}
					}
				\end{documentationx}

		\noindent\makebox[\linewidth]{\rule{.7\paperwidth}{0.4pt}}

			\begin{redbox}
				Bestimmte Befehle -- wie \zb die \verb|itemize|-Umgebung -- verursachen hier Probleme. Sie müssen mit dem Schlüsselwort \verb|\protect| geschützt werden:
				\begin{verbatim}
					\protect\begin{itemize}
					    \protect\item Lorem ipsum
					    \protect\item dolor sit amet.
					\protect\end{itemize}
				\end{verbatim}
			\end{redbox}

		\section{Abstände}
			\label{abstaende beweis beispiel}
			Wenn die \texttt{beweis}- und den \texttt{beispiel(e)}-Umgebungen mit Aufzählungen beginnen bzw. enden, wird sowohl oben als auch unten ein zu großer Abstand eingefügt:

			\begin{verbatim}
				\begin{beweis}[für P $\not=$ NP]
				    \begin{itemize}
				        \item Text
				        \item Text
				     \end{itemize}
				\end{beweis}
			\end{verbatim}
			\fbox{
				\begin{minipage}[t]{.9\textwidth}
					\bigskip
					\begin{beweis}[für P $\not=$ NP]
						\begin{itemize}
							\item Text
							\item Text
						\end{itemize}
					\end{beweis}
					\bigskip
				\end{minipage}
			}

			\vspace*{1em}


			Der zweite -- ebenfalls optionale -- Parameter beeinflusst die Abstände oben und unten:
			\begin{itemize}
				\item \texttt o sorgt dafür, dass \texttt{itemize}- und ähnliche Umgebungen den richtigen Abstand nach oben haben;
				\item \texttt u sorgt dafür, dass das $\Box$-Symbol auf richtiger Höhe ist (betrifft ebenfalls \texttt{itemize}, jedoch natürlich nur Beweise -- der Befehl existiert trotzdem auch für die \texttt{beispiel}-Umgebung);
				\item \texttt b schaltet beides ein.
			\end{itemize}


			\begin{verbatim}
				\begin{beweis}[für P $\not=$ NP][b]
				    \begin{itemize}
				        \item Text
				        \item Text
				    \end{itemize}
				\end{beweis}
			\end{verbatim}

			\fbox{
				\begin{minipage}[t]{.9\textwidth}
					\bigskip
					\begin{beweis}[für P $\not=$ NP][b]
						\begin{itemize}
							\item Text
							\item Text
						\end{itemize}
					\end{beweis}
					\bigskip
				\end{minipage}
			}

	\chapter{Farbige Boxen}
		\section{Standardboxen (blau, grün, gelb)}
			Es gibt blaue, grüne und gelbe Boxen. Sie werden alle auf die gleiche Weise benutzt. Sie wurden als Umgebungen namens \verb|blueboxIdx|, \verb|yellowboxIdx| und \verb|greenboxIdx| definiert und besitzen zwei Parameter, wobei der erste optional ist:
			\begin{itemize}
				\item Der erste Parameter ist ein Label, auf den mit \verb|\ref| oder \verb|\vref| zugegriffen werden kann.
				\item Der zweite Parameter ist der Titel der Box. Auf einen fortlaufenden Zähler kann mit \verb|\boxnummer| zugegriffen werden.
			\end{itemize}
			
			\begin{verbatim}
				\begin{blueboxIdx}[TM-def]{Begriff \boxnummer: Turingmaschine}
				     Eine \neuerbegriff{Turingmaschine} ist ein wichtiges Rechnermodell
				     der Theoretischen Informatik. Eine Turingmaschine modelliert die
				     Arbeitsweise eines Computers auf besonders einfache und
				     mathematisch gut zu analysierende Weise. Sie ist benannt nach dem
				     Mathematiker \name{Alan Turing}, der sie 1936 einführte.
				\end{blueboxIdx}

				[\ellipse]
				Die Turingmaschine wurde in Box \vref{TM-def} erläutert.
			\end{verbatim}
			
			\fbox{
				\begin{minipage}[t]{.9\textwidth}
					\bigskip
					\begin{blueboxIdx}[TM-def]{Begriff \boxnummer: Turingmaschine}
						Eine \neuerbegriff{Turingmaschine} ist ein wichtiges Rechnermodell der Theoretischen Informatik. Eine Turingmaschine modelliert die Arbeitsweise eines Computers auf besonders einfache und mathematisch gut zu analysierende Weise. Sie ist benannt nach dem Mathematiker \name{Alan Turing}, der sie 1936 einführte.
					\end{blueboxIdx}

					[\ellipse]
					Die Turingmaschine wurde in Box \vref{TM-def} erläutert.
					\bigskip
				\end{minipage}
			}
		\newpage
		\section{Warnboxen (rot)}
			Rote Boxen (als \verb|redbox|-Umgebung definiert) haben als Titel stets \texttt{Achtung!}. Sie besitzen keinen eigenen Index.

			\begin{verbatim}
				\begin{redbox}
				     Dieser Satz ist wichtig.
				\end{redbox}
			\end{verbatim}

			\fbox{
				\begin{minipage}[t]{.9\textwidth}
					\bigskip
					\begin{redbox}
						Dieser Satz ist wichtig.
					\end{redbox}
					\bigskip
				\end{minipage}
			}

	\chapter{Weitere nützliche Anmerkungen und Codebeispiele}
		\section{Codeschnipsel}
			Dieses Kapitel listet ein paar Codeschnipsel auf, die beim täglichen Gebrauch von \LaTeX\ helfen können.
			\begin{description}
				\item[Bedeutung eines Makros:]
					\verb|\meaning\<makroname>|. Beispielsweise liefert\\\verb|\meaning\makeatletter| Folgendes: \texttt{\meaning\makeatletter}
				\item[Links setzen:]
					\verb|\href{URL}{text}|: Beispielsweise ein Link \href{https://de.wikipedia.org/wiki/LaTeX}{zu Wikipedia}.
			\end{description}

		\section{Makroprogrammierung in ${\mathrm{T\!_{\displaystyle E}\!X}}$}
			Der folgende Code implementiert ein einfaches Array:
			\begin{verbatim}
\newcounter{foo}

\makeatletter
    \def\pisarzy@i 	{Krzysztof Kamil Baczyński}
    \def\pisarzy@ii	{Stanisław Barańczak}
    \def\pisarzy@iii{Anna Brzezińska}
    \def\pisarzy@iv {Ewa Lipska}
\makeatother

\def\pisarzy[#1]{%                            <-- Anmerkung
    \setcounter{foo}{#1}%
    \item %
    \makeatletter %
    \csname pisarzy@\roman{foo}\endcsname%
    \makeatother %
}
			\end{verbatim}
			\newcounter{foo}

			\makeatletter
				\def\pisarzy@i 	{Krzysztof Kamil Baczyński}
				\def\pisarzy@ii	{Stanisław Barańczak}
				\def\pisarzy@iii{Anna Brzezińska}
				\def\pisarzy@iv {Ewa Lipska}
			\makeatother

			\def\pisarzy[#1]{%
				\setcounter{foo}{#1}%
				\item %
				\makeatletter %
				\csname pisarzy@\roman{foo}\endcsname%
				\makeatother %
			}

			\newpage
			So greift man auf die Felder zu:

			\begin{verbatim}
\begin{itemize}
    \pisarzy[1]
    \pisarzy[2]
    \pisarzy[3]
    \pisarzy[4]
\end{itemize}
			\end{verbatim}

			\begin{itemize}
				\pisarzy[1]
				\pisarzy[2]
				\pisarzy[3]
				\pisarzy[4]
			\end{itemize}

			\emph{Anmerkung:} Man könnte hier auch \verb|\def\pisarzy--#1--]| schreiben, dann greift man auf die Felder mit \verb|\pisarzy --42--]| zu.

			Basierend auf dem \href{http://pgfplots.sourceforge.net/TeX-programming-notes.pdf}{Skript von C. Feuersänger}.
\end{document}
